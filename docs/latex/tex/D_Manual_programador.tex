\apendice{Documentación técnica de programación}

\section{Introducción}
Este apéndice aborda la documentación técnica de programación de la aplicación. En las siguientes secciones se detallan diferentes detalles importantes para trabajar en el proyecto, como pueden ser:
\begin{enumerate}
\tightlist

\item Estructura de directorios.
\item Instalación del entorno de desarrollo.
\item Obtención del código fuente.
\item Ejecución de la aplicación.
\item Pruebas realizadas.

\end{enumerate}

\section{Estructura de directorios}

La estructura de directorios de la aplicación es la siguiente:

\begin{itemize}
\tightlist

\item \texttt{/}: Directorio raíz del proyecto. Contiene el archivo \texttt{requirements.txt} con la información de las dependencias del proyecto, el archivo \texttt{README.md} con información relevante sobre el proyecto y el archivo \texttt{LICENSE} con la licencia del proyecto.

\item \texttt{/app/}: Directorio con el código fuente de la aplicación. Además incluye en fichero \texttt{\textunderscore\textunderscore init\textunderscore\textunderscore .py} que lanza y define las rutas en la aplicación web utilizando \texttt{Flask}.

\item \texttt{/app/models/}: Directorio donde se almacenan los modelos de \emph{machine learning} entrenados que la aplicación utiliza para realizar las predicciones de los estilos musicales.

\item \texttt{/app/src/}: Directorio donde se encuentra el código fuente Python. Aquí es donde se realiza todo el procesamiento interno de la aplicación.

\item \texttt{/app/static/}: Directorio donde se almacenan los archivos estáticos utilizados en la interfaz web de la aplicación. Por ejemplo, archivos CSS, pistas de audio subidas o scripts de JavaScript.

\item \texttt{/app/templates/}: Directorio donde se almacenan los archivos HTML que la aplicación utiliza para generar las páginas web.

\item \texttt{/app/tests/}: Directorio donde se almacenan los tests unitarios y de integración utilizados para verificar el correcto funcionamiento de la aplicación.

\item \texttt{/data/}: Directorio donde se almacenan los datos utilizados por la aplicación para realizar el entrenamiento o el procesado.

\item \texttt{/data/processed/}: Directorio donde se almacenan los datos procesados y listos para ser entrenados en el modelo de machine learning.

\item \texttt{/data/raw/}: Directorio donde se almacenan los ficheros de audio sin procesar.

\item \texttt{/docs/latex/}: Documentación del proyecto en formato \LaTeX.

\item \texttt{/examples/}: Ejemplos de uso de las bibliotecas.

\end{itemize}

\section{Manual del programador}

Esta sección tiene como objetivo ser una guía para que los futuros programadores puedan entender el código fuente, configurar el entorno de desarrollo y poder contribuir en el proyecto.

Para el desarrollo del proyecto se utilizó Visual Studio Code con el lenguaje de programación Python. Además, se utilizó Git para el sistema de control de versiones. Se recomienda tener ciertos conocimientos en estos sistemas antes de empezar.

Asimismo, para el diseño de la interfaz de usuario, se utilizó el lenguaje de marcado \texttt{HTML}, que proporciona la estructura básica de las páginas web. Para personalizar la apariencia de estas páginas, se aplicaron estilos mediante \texttt{CSS} y se implementaron funcionalidades interactivas a partir de \texttt{JavaScript}. Complementariamente, LaTeX se utilizó para la elaboración de la documentación, gracias a su capacidad para gestionar referencias y presentar ecuaciones y algoritmos de manera clara.

\subsection{Herramientas recomendadas}

Para contribuir al desarrollo del proyecto se recomienda utilizar las siguientes herramientas.

\begin{itemize}
\tightlist

\item \textbf{Git}: Se trata del sistema de control de versiones utilizado para realizar un seguimiento de los cambios en el código fuente de la aplicación a lo largo del tiempo. Se puede combinar con GitHub. \cite{Git, GitHub}

\imagen{Git_console}{Consola de Git.}

\item \textbf{Visual Studio Code}: Es el editor de código fuente recomendado para continuar con el desarrollo del proyecto. Se trata de un editor de código simple y ligero pero con una gran cantidad de extensiones que facilitan el desarrollo.
Por ejemplo, incluye \emph{linting} de código, formateo automático del código y soporte con Jupyter Notebooks o Docker \cite{Microsoft_2021}.

\imagen{VSCode}{Visual Studio Code.}

\item \textbf{Jupyter notebooks}: Herramienta de desarrollo de código y visualizaciones. Es una herramienta muy útil para realizar experimentación ya que une la ejecución de código fuente con la visualización de datos y la documentación.
No se trata de una herramienta para realizar una versión final del producto, pero es muy útil para probar y generar prototipos de una forma rápida \cite{Jupyter}.

\imagen{Jupyter_notebook}{Jupyter Notebook.}

\end{itemize}

\section{Compilación, instalación y ejecución del proyecto}

Para obtener el código fuente del proyecto los pasos a seguir son los siguientes:

\begin{itemize}
\tightlist

\item \textbf{Clonar el repositorio Git}
	\begin{itemize}
	\tightlist
		\item \texttt{git clone https://github.com/jle1001/AI-Music.git}
	\end{itemize}

\item \textbf{Crear un entorno virtual de Python}: 
	\begin{itemize}
	\tightlist
		\item \texttt{python3 -m venv venv}
	\end{itemize}

\item \textbf{Activar el entorno virtual}: 
	\begin{itemize}
		\item Linux: \texttt{source venv/bin/activate}
		
		\item Windows: \texttt{venv\textbackslash Scripts\textbackslash activate}
	\end{itemize}

\item \textbf{Instalación de dependencias}
	\begin{itemize}
	\tightlist
		\item \texttt{pip install -r requirements.txt}
	\end{itemize}

\item \textbf{Iniciar el servidor}
	\begin{itemize}
	\tightlist
		\item \texttt{flask run --debug}
	\end{itemize}
\end{itemize}

Una vez que el servidor está en funcionamiento, se puede acceder a la aplicación web a través de un navegador entrando a la dirección \texttt{localhost:5000}.

\section{Pruebas del sistema}

\subsection{Pruebas unitarias}

Las pruebas unitarias son un tipo de pruebas de software en las que se comprueban individualmente las partes más pequeñas del sistema. Estas partes pueden ser funciones, métodos o clases.
En este tipo de pruebas no se verifica el correcto funcionamiento del modelo de predicción, sino el funcionamiento de cada parte del código que se necesita para llegar hasta ahí. Por ejemplo:

\begin{itemize}
\tightlist

\item \textbf{Pruebas unitarias en la generación de predicciones}: verificación de que el uso del modelo entrenado genera una cantidad correcta de predicciones. Por ejemplo, se puede verificar que la capa de salida del modelo genera una cantidad de clases igual al número de géneros musicales en el conjunto de datos.

\imagen{number_predictions_unit_test}{Código del test unitario que comprueba el número correcto de generación de predicciones.}

\imagen{number_predictions_unit_test_OK}{Resultado de la ejecución del test unitario.}

\end{itemize}

\subsection{Pruebas de interfaz de usuario}

Las pruebas de interfaz de usuario son aquellas que se utilizan para verificar que la interacción entre el usuario y el software se realiza correctamente. Por ejemplo:

\begin{itemize}
\tightlist

\item \textbf{Prueba de carga de archivos}: esta prueba verifica que la aplicación puede cargar correctamente los archivos deseados por el usuario.
	\begin{enumerate}
	\tightlist

	\item \textbf{El usuario carga el archivo}: \texttt{06. Give.mp3}
	\item \textbf{Resultado esperado}: se muestra en pantalla información básica sobre el fichero de audio, un menú de selección de modelos para realizar la predicción y un reproductor para escuchar el fichero de audio.
	\imagen{upload_track_test}{Resultado obtenido. Es correcto.}

	\end{enumerate}
\item \textbf{Prueba de visualización de resultados}: esta prueba verifica que, una vez realizada la predicción, los resultados se muestran correctamente para el usuario.
	\begin{enumerate}
	\tightlist

	\item \textbf{El usuario presiona el botón \textit{Submit}}
	\item \textbf{Resultado esperado}: se muestra en pantalla una lista de tres predicciones con su probabilidad asociada, además de un menú interactivo donde se pueden ver distintas visualizaciones de las características del fichero de audio en concreto
	\imagen{analysis_track_test}{Resultado obtenido. Es correcto.}

	\end{enumerate}
\end{itemize}