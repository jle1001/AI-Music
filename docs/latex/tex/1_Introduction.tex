\capitulo{1}{Introduction}

\section{Context}
Music is a culturally significant and important element, with a wide variety of styles and genres.
The automated identification of musical styles is an emerging field of study, aiming to create intelligent systems capable of automatically recognizing and classifying the musical style of a song or music piece without any human intervention.

This field of study has important applications in areas such as music recommendation or digital audio library organization.

The application of artificial intelligence algorithms has shown considerable success in this area.

The present work aims to develop an automatic music style recognition system using audio processing and artificial intelligence algorithms.

\section{Project Structure}
The present document follows the next structure:

\begin{itemize}
\item \textbf{Introduction}: context and description of the problem,  project structure and relevant links.
\item \textbf{Project objectives}: explanation of project's main objectives.
\item \textbf{Theoretical concepts}: short explanation of the main theoretical concepts that are necessary to understand the problem addressed in the project
\item \textbf{Techniques and tools}: present the methodology, techniques and development tools used in the project's development.
\item \textbf{Relevant aspects of project development}: important aspects about the project development.
\item \textbf{Related works}: state of the art and related works in the area.
\item \textbf{Conclusions and future works}: conclusions obtained after the completion of the project and possible future lines of work.
\end{itemize}

In addition, \textit{Annexes} document contains the following structure:

\begin{itemize}
\item \textbf{Appendix A. Manuals}:
\item \textbf{Appendix B. Requirements Specification}:
\item \textbf{Appendix C. Design Specification}:
\item \textbf{Appendix D. Technical Documentation}:
\item \textbf{Appendix E. User Documentation}:
\end{itemize}