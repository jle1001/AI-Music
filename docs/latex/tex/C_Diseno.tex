\apendice{Especificación de diseño}

\section{Introducción}

En esta sección se define cómo se han implementado las especificaciones técnicas vistas en el apéndice B.

\section{Diseño de datos}

La aplicación trabaja con distintos tipos de datos.

\begin{itemize}
\tightlist

\item \textbf{Archivos CSV}:

\item \textbf{Archivos de audio}:

\item \textbf{Archivos pickle}:

\end{itemize}

INSERTAR DIAGRAMA DE CLASES, RELACIONES, E/R, ETC.

\section{Diseño procedimental}

DISEÑO PROCEDIMENTAL SIGNIFICA CÓMO SE HAN DISEÑADO LOS PROCEDIMIENTOS DE LA APLICACIÓN (CÓMO LOS MÉTODOS, FUNCIONES, ETC.)
MOSTRAR CADA PROCEDIMIENTO EN PSEUDOCODIGO O LENGUAJE NATURAL.
diagramas de interacción, secuencia, etc.

\section{Diseño arquitectónico}

MODELO ARQUITECTONICO PREVISTO, PATRONES DE DISEÑO. Por ejemplo decir si hemos usado patrón vista controlador, patrón MVM, etc.
herencias, genericidad, clases, etc.

EXPLICAR LA FORMA GENERAL DE LA ARQUITECTURA DE LA APLICACIÓN (CON DIAGRAMAS, FLECHAS, ETC.)

INSERTAR NUEVA SECCIÓN CON DISEÑO DE INTERFACES SI PROCEDE.