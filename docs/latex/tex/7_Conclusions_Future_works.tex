\capitulo{7}{Conclusions and future works}

\section{Conclusions}
Through this project, the performance but also the complexity of using Artificial Intelligence (AI) for the automatic classification of musical styles has been proved.

In general terms, the system has proven to be reasonably effective in the task of classifying musical styles. However, limitations have also been identified in the process.

First, although the audio feature extraction process plays a crucial role in the development and training of musical style classification models, it might be worth considering to explore other methods or implementing more advanced techniques in order to improve the quality of the data that feeds the neural network.

It may be beneficial explore deep learning techniques such as transfer learning. Pre-trained models could be used as a starting point, which could improve the quality of the model.

The size of the dataset has a direct impact on the effectiveness of the system. Using a larger and more varied dataset could improve the system's ability to generalize.

\textbf{In conclusion}, this project has served to demonstrate that music style classification using AI is feasible and that artificial intelligence applied to music offers great potential for future research.

\section{Future works}

\begin{enumerate}
\item \textbf{Experimentation with different models}: Convolutional neural networks have been mainly used in this project, but there are many other network architectures. For example, recurrent neural networks (RNNs) or generative adversarial models (GANs) could be fine choices for experimentation.

\item \textbf{Explore semi-supervised and unsupervised learning approaches}: Given the difficulty of labeling large music datasets, semi-supervised or unsupervised learning techniques could be considered. These approaches could be useful to take advantage of unlabeled data and improve system performance.

\item \textbf{User feedback}: Allow users to correct model incorrect classifications and use that information to improve the model.

\item \textbf{Build a recommendation model}: With the music style classification model, it could be interesting to build a music recommendation system that suggests songs to the users, based on their preferences and listening habits.
\end{enumerate}

TODO: ADD MORE FUTURE ENHANCEMENTS AND IMPROVE THIS SECTION!