\apendice{Plan de Proyecto Software}

\section{Introducción}

Este plan de proyecto de software se refiere al desarrollo de una aplicación que clasifica automáticamente los géneros musicales basándose en características de audio. 
Este proyecto ha sido concebido como un trabajo de fin de grado y ha implicado la implementación de técnicas de aprendizaje automático y procesamiento de señales de audio digitales para lograr el objetivo.
La aplicación busca proporcionar una solución efectiva para clasificar pistas musicales de forma automática.

En este apéndice se describirá la planificación sobre la que se ha desarrollado el proyecto.

\section{Planificación temporal}

El desarrollo del proyecto se organizó siguiendo la metodología ágil de SCRUM, con iteraciones semanales o "sprints". 
Cada sprint implicó una serie de actividades que culminaron con una entrega incremental del proyecto.

\textbf{Sprints totales}: 14

\subsection{Sprint 1 (08/03/2023 - 15/03/2023)}

\subsection{Sprint 2 (15/03/2023 - 22/03/2023)}

\subsection{Sprint 3 (22/03/2023 - 05/04/2023)}

\subsection{Sprint 4 (05/04/2023 - 12/04/2023)}

\subsection{Sprint 5 (12/04/2023 - 19/04/2023)}

\subsection{Sprint 6 (19/04/2023 - 26/04/2023)}

\subsection{Sprint 7 (26/04/2023 - 10/05/2023)}

\subsection{Sprint 8 (10/05/2023 - 17/05/2023)}

\subsection{Sprint 9 (17/05/2023 - 24/05/2023)}

\subsection{Sprint 10 (24/05/2023 - 02/06/2023)}

\subsection{Sprint 11 (02/06/2023 - 09/06/2023)}

\subsection{Sprint 12 (09/06/2023 - 16/06/2023)}

\subsection{Sprint 13 (16/06/2023 - 23/06/2023)}

\subsection{Sprint 14 (23/06/2023 - 30/06/2023)}

\section{Estudio de viabilidad}

\subsection{Viabilidad económica}

\subsection{Viabilidad legal}