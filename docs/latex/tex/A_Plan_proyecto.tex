\apendice{Plan de Proyecto Software}

\section{Introducción}

Este plan de proyecto de software se refiere al desarrollo de una aplicación que clasifica automáticamente los géneros musicales basándose en características de audio. 
Este proyecto ha sido concebido como un trabajo de fin de grado y ha implicado la implementación de técnicas de aprendizaje automático y procesamiento de señales de audio digitales para lograr el objetivo.
La aplicación busca proporcionar una solución efectiva para clasificar pistas musicales de forma automática.

En este apéndice se describirá la planificación sobre la que se ha desarrollado el proyecto.

\section{Planificación temporal}

El desarrollo del proyecto se organizó siguiendo la metodología ágil de SCRUM, con iteraciones semanales o "<sprints">. 
Cada sprint implicó una serie de actividades que culminaron con una entrega incremental del proyecto.

\textbf{Sprints totales}: 14

\subsection{Sprint 1 (08/03/2023 - 15/03/2023)}

En el Sprint 1, se realizó la configuración inicial del proyecto, añadiendo una descripción detallada en el archivo \texttt{README.md} y estableciendo las bases de la documentación. Asimismo, se implementaron componentes básicos y se trabajó en el diseño de la aplicación, aprovechando las funcionalidades de la biblioteca \textit{librosa} para el procesamiento de audio.

\subsection{Sprint 2 (15/03/2023 - 22/03/2023)}

En el Sprint 2, se configuró el entorno de desarrollo utilizando Poetry, que posteriormente se descartó ante el uso de entornos virtuales clásicos, y se implementó el módulo \texttt{train\textunderscore classifier} para entrenar el clasificador de géneros musicales. De igual manera, también se llevó a cabo la extracción de características MFCC y se realizó un preprocesamiento básico de los datos.

\subsection{Sprint 3 (22/03/2023 - 05/04/2023)}

En el Sprint 3, se realizaron pequeños cambios en el código e investigación sobre el área.

\subsection{Sprint 4 (05/04/2023 - 12/04/2023)}

En el Sprint 4, se realizaron diversas tareas. En primer lugar, se agregó un archivo \texttt{requirements.txt} para gestionar las dependencias del proyecto y se creó la estructura inicial de Flask para desarrollar la aplicación web. Complementariamente, se diseñó una aplicación Flask básica y se llevó a cabo el preprocesamiento de géneros musicales. Además, se realizaron ajustes en la estructura del proyecto y se actualizó la extracción de características MFCC. También se implementó la visualización de características de audio, como la forma de onda, el espectrograma, el cromagrama y los MFCC. Finalmente, se realizaron mejoras en el estilo y se añadió un modelo de aprendizaje automático inicial.

\subsection{Sprint 5 (12/04/2023 - 19/04/2023)}

En el Sprint 5, se trabajó en los aspectos visuales del proyecto. Se crearon mockups e imágenes de prueba para la interfaz de usuario, permitiendo una mejor comprensión de cómo se presentaría la aplicación.

\subsection{Sprint 6 (19/04/2023 - 26/04/2023)}

En el Sprint 6, se avanzó en la sección de carga del proyecto y se añadió una página de inicio. Además, se implementó un reproductor de música que permitía a los usuarios escuchar las pistas seleccionadas.

\subsection{Sprint 7 (26/04/2023 - 10/05/2023)}

En el Sprint 7, se llevaron a cabo varias tareas importantes. Se añadió un nuevo modelo al sistema y se actualizó la documentación correspondiente. Paralelamente, también se trabajó en la implementación de las páginas de registro e inicio de sesión para los usuarios, aunque finalmente no pudieron ser llevadas a cabo. También se realizaron ajustes en el guardado del modelo y se añadieron las páginas de inicio de sesión y registro al índice del proyecto.

\subsection{Sprint 8 (10/05/2023 - 17/05/2023)}

En el Sprint 8, se dedicó tiempo a la búsqueda de referencias y documentación relevante para el proyecto. Se recopilaron trabajos relacionados, papers e información general que proporcionaron una base sólida para la consolidación de la aplicación.

\subsection{Sprint 9 (17/05/2023 - 24/05/2023)}

En el Sprint 9, se realizaron mejoras en la extracción de características y se llevó a cabo un proceso de limpieza, eliminando archivos de borrador que ya no eran necesarios.

\subsection{Sprint 10 (24/05/2023 - 02/06/2023)}

En el Sprint 10, se añadió documentación adicional para complementar el proyecto, asegurando que todos los aspectos importantes estuvieran debidamente explicados.

\subsection{Sprint 11 (02/06/2023 - 09/06/2023)}

En el Sprint 11, se implementaron generadores para la fase de entrenamiento del modelo, lo que permitió una mejor gestión de los datos. También se actualizó la representación de géneros predichos.

\subsection{Sprint 12 (09/06/2023 - 16/06/2023)}

En el Sprint 12, se realizaron cambios en el diseño de la aplicación, mejorando la representación visual de las características de audio. Se llevaron a cabo actualizaciones en el diseño de los colores utilizados en la interfaz.

\subsection{Sprint 13 (16/06/2023 - 23/06/2023)}

En el Sprint 13, se realizaron varios cambios significativos. Por un lado se implementaron diferentes mejoras en el procesamiento y entrenamiento de los datos. De igual modo, se actualizaron el reproductor de audio y la información de las pistas. Además, se realizaron múltiples actualizaciones en la memoria, los anexos y la aplicación web. Eventualmente, se llevaron a cabo correcciones menores de estilo para mejorar la presentación general.

\subsection{Sprint 14 (23/06/2023 - 05/06/2023)}

En el Sprint 14, se continuaron realizando actualizaciones y mejoras en la documentación. Se realizaron cambios específicos en la sección B de los anexos, relacionada con los requisitos funcionales y no funcionales. También se actualizaron las secciones A, D y E de los anexos, que incluyen el plan del proyecto, el manual del programador y el manual de usuario, respectivamente. Se agregó un diagrama general de casos de uso y se llevaron a cabo actualizaciones adicionales en la documentación, la aplicación web y los anexos. Además se desplegó la aplicación web en un servidor en la nube, concretamente en la plataforma \texttt{PythonAnywhere}.

\section{Estudio de viabilidad}

\subsection{Viabilidad económica}

La viabilidad económica de un proyecto de software se refiere a la capacidad para generar ingresos tanto para cubrir el desarrollo como para proporcionar una rentabilidad a medio y largo plazo. El presente proyecto no está planteado con un objetivo económico, por lo que esta sección se va a dedicar a explorar un posible plan de monetización de forma ficticia, explorando los costes totales de desarrollo de software y potenciales métodos de monetización.

\subsection{Costes de hardware}
Los costes de hardware se refieren al gasto económico que se realiza para obtener los diferentes elementos hardware que se necesitan para poder llevar a cabo el desarrollo del proyecto. Los elementos hardware pueden ser ordenadores, periféricos o componentes internos por ejemplo.
En este caso se ha planteado la compra de un ordenador portátil de unos 1000€. Esta decisión se toma teníendo en cuenta los siguientes factores:

\begin{itemize}
\tightlist
\item \textbf{Rendimiento}: Este proyecto requiere un uso de algoritmos de aprendizaje automático y procesamiento de datos, por lo que se demanda una cierta capacidad de computación. 
Un equipo dentro del rango de precio planteado viene equipado con un procesador potente, como un Intel Core i7 (12th o 13th Gen) o un AMD Ryzen 7 (6th Gen). Además suelen incluir una tarjeta gráfica dedicada, como una NVIDIA RTX 3060 con 6 GB de VRam, con la ventaja en rendimiento que esto conlleva en el uso de bibliotecas de \textit{Deep Learning} como TensorFlow.

\item \textbf{Durabilidad}: Los ordenadores de este rango de precio suelen contar con componentes de mayor calidad asegurando una mayor vida útil. La durabilidad es importante en proyectos como este ya que un fallo de hardware podría tener un impacto significativo en la productividad.

\item \textbf{Soporte}: Los fabricantes proporcionan un soporte de mayor calidad en ordenadores situados en este rango de precio. Este soporte puede incluir mejores condiciones de garantía o actualizaciones más duraderas en partes vitales del sistema como la BIOS.
\end{itemize}

\begin{table}[h]
\centering
\begin{tabular}{|l|c|}
\hline
\textbf{Concepto} & \textbf{Coste} \\ 
\hline
Ordenador portátil & 1000€ \\ 
\hline
Total & 1000€ \\ 
\hline
\end{tabular}
\caption{Costes de hardware}
\end{table}

\subsection{Costes de software}
Los costes de software se refieren al gasto económico que se realiza en la adquisición de las distintas licencias de software que son necesarias para la realización del proyecto.

\begin{itemize}
\tightlist
\item \textbf{Sistema Operativo}: Se ha elegido el sistema operativo Windows 11 Home como base para realizar el desarrollo del proyecto. Se podrían plantear alternativas gratuitas como alguna distribución de Linux pero por sencillez y extensión de uso se ha elegido Windows.

\item \textbf{Servicios en la nube}: Durante las etapas intensivas de entrenamiento de modelos y almacenamiento de datos, podría llegar a ser necesario el uso de servicios de computación en la nube como AWS \cite{AWS} o Microsoft Azure \cite{Kcpitt}. El coste puede variar según el uso necesario. Por ejemplo en un proyecto de estas características, contando con unos 100 GB de datos en AWS, se estimaría un costo de almacenamiento de unos 2€/mensuales y un costo de procesamiento de unos 30€/mensuales, utilizando 10 horas en una instancia \texttt{p3.2xlarge}, la cual es comúnmente utilizada para labores de aprendizaje automático. En total, el coste de la computación en la nube a lo largo del proyecto ha sido de 192€.
\end{itemize}

\begin{table}[h]
\centering
\begin{tabular}{|l|c|}
\hline
\textbf{Concepto} & \textbf{Coste} \\
\hline
Windows 11 Home \cite{win11} & 145€ \\
Computación en la nube & 192€ \\
\hline
\textbf{Total} & \textbf{337€} \\
\hline
\end{tabular}
\caption{Costes de software para el desarrollo del proyecto}
\end{table}

\subsection{Otros costes}

\begin{table}[h]
\centering
\begin{tabular}{|l|c|}
\hline
\textbf{Concepto} & \textbf{Coste} \\
\hline
Electricidad (mensual) & 50€ \\
Internet (mensual) & 35€ \\
Espacio de trabajo (mensual) & 350€ \\
\hline
\textbf{Total (mensual)} & \textbf{435€} \\
\hline
\textbf{Total (6 meses)} & \textbf{2610€} \\
\hline
\end{tabular}
\caption{Otros costes para el desarrollo del proyecto}
\end{table}

\subsection{Costes totales}
Teniendo en cuenta una duración de desarrollo de proyecto de 6 meses, el coste total ha sido:

\begin{table}[h]
\centering
\begin{tabular}{|l|c|}
\hline
\textbf{Concepto} & \textbf{Coste} \\
\hline
Costes de hardware & 50€ \\
Costes de software & 32€ \\
Otros costes & 2610€ \\
\hline
\textbf{Total} & \textbf{2692€} \\
\hline
\end{tabular}
\caption{Costes totales de desarrollo del proyecto}
\end{table}

\subsection{Ingresos}
La idea del proyecto es generar ingresos hasta el punto de rentabilizar el gasto de desarrollo como mínimo. 
Existen diversas formas en las que la aplicación podría generar ingresos como por ejemplo:

\begin{itemize}
\tightlist

\item \textbf{Sistema de suscripción}: Se puede ofrecer la aplicación con distintas características según el \textit{tier} o el tipo de suscripción. Podrían existir tres \textit{tier}:
	\begin{itemize}
	\tightlist
	\item \textbf{Suscripción gratuita}: límite 5 canciones/diarias

	\item \textbf{Suscripción silver 5€/mes}: canciones ilimitadas.

	\item \textbf{Suscripción gold 15€/mes}: canciones ilimitadas y análisis con más detalle.
	\end{itemize}

\item \textbf{Publicidad}: Se puede monetizar el servicio a través de anuncios. Los usuarios que usen la versión gratuita de la aplicación verán anuncios en distintas partes de la interfaz, mientras que si pagan una mensualidad (5€/mes) tiene la opción de eliminar anuncios.

\item \textbf{Integraciones de API}: Se puede ofrecer un servicio de ofrecer la API a aplicaciones de terceros para utilizar la aplicación. Esta API puede ser gratuita, con limitaciones en el número de canciones a predecir, o de pago, con un menor número de limitaciones.
\end{itemize}

\subsection{Viabilidad legal}

La viabilidad legal del proyecto se refiere al cumplimiento de diversas leyes y obligaciones a la hora de desarrollar el software.

\subsection{Derechos de autor}

El tema de los derechos de autor en el mundo de la música es un tema complicado. Los derechos de autor de la música son un área legal compleja que envuelve partes como artistas, productores y discografías. Este proyecto va a utilizar un conjunto de datos musicales libres de derechos de autor lo que facilita el trabajo eliminando esta parte del proceso de desarrollo.

A la hora de la subida de archivos musicales para su clasificación existe un problema, el usuario final puede subir cualquier fichero de audio, tenga derechos de autor o no. Por lo tanto, el procedimiento a realizar será la eliminación de cualquier fichero musical de la aplicación tras realizar el proceso de predicción.

\subsection{Licencias de software}

Según el software utilizado, se tienen en cuenta las siguientes licencias: \cite{Creative_Commons, Open_Source_Initiative_2023}

\begin{table}[h]
\centering
\caption{Licencias de las herramientas utilizadas}
\begin{tabular}{|l|l|}
\hline
\textbf{Herramienta} & \textbf{Licencia} \\
\hline
Python \cite{3.10_Documentation} & PSF \\
Librosa \cite{librosa} & ISC \\
TensorFlow \cite{TensorFlow} & Apache 2.0 \\
ScikitLearn \cite{scikit} & BSD-3-Clause \\
Flask \cite{Flask} & BSD-3-Clause \\
FMA Dataset \cite{fma_dataset} & CC BY-NC-SA 4.0 \\
Matplotlib \cite{matplotlib} & PSF \\
Pandas \cite{pandas} & BSD-3-Clause \\
Numpy \cite{NumPy} & BSD-3-Clause \\
\hline
\end{tabular}
\end{table}

Según las herramientas utilizadas, se ha elegido la licencia MIT para la distribución y uso del software desarrollado. Esta es una licencia de código abierto permisiva, la cual permite a los usuarios utilizar, modificar y distribuir el software bajo ciertas condiciones, requiriendo la inclusión del aviso de derechos de autor y la renuncia de responsabilidad en el software distribuido.
Una de las características principales de la licencia MIT es su flexibilidad. Permite a los usuarios utilizar el software con pocas restricciones y sin imponer condiciones adicionales. Además, ofrece libertad para modificar y distribuir el software tanto en proyectos de código abierto como en proyectos comerciales.
La elección de la licencia MIT se basa en su adecuación a las necesidades y especificidades de este proyecto. Algunas razones por las cuales es la licencia que mejor se ajusta son:

\begin{enumerate}
\tightlist

\item \textbf{Flexibilidad y colaboración}: La licencia MIT \cite{Open_Source_Initiative_2023} permite que otras personas utilicen, modifiquen y distribuyan el software sin imponer restricciones excesivas. Esto fomenta la colaboración y la participación de la comunidad, lo cual es beneficioso para un proyecto de desarrollo de software.
\item \textbf{Compatibilidad}: La licencia MIT es compatible con la mayoría de las otras licencias de código abierto. Esto es importante, ya que el proyecto utiliza diversas herramientas con diferentes licencias, asegurando que no haya conflictos y permitiendo una integración fluida entre ellas.
\item \textbf{Reconocimiento y protección de derechos de autor}: La licencia MIT incluye disposiciones que protegen los derechos de autor y exigen la inclusión del aviso de derechos de autor en el software distribuido. Esto garantiza el reconocimiento adecuado del autor original del proyecto y protege sus derechos.

\end{enumerate}