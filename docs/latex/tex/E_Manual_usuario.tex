\apendice{Documentación de usuario}

\section{Introducción}

Esta sección proporciona instrucciones de uso detalladas sobre cómo utilizar la aplicación, desde el proceso de instalación hasta el proceso de subida de archivos.

\section{Requisitos de usuarios}

\subsection{Entorno local}

\begin{table}[h]
	\centering
	\begin{tabular}{|c|c|}
		\hline
		\textbf{Requisito} & \textbf{Especificación}\\
		\hline
		CPU & 2.0 GHz o superior\\
		\hline
		RAM & 4GB o superior\\
		\hline
		Disco duro & 2GB de almacenamiento libre\\
		\hline
		Sistema Operativo & Compatible con Python 3.10 o superior\\
		\hline
		Python & Versión 3.10 o superior\\
		\hline
		Bibliotecas de Python & Especificadas en la sección de instalación\\
		\hline
	\end{tabular}
	\caption{Requisitos para el Funcionamiento Local del Sistema}
\end{table}

\begin{itemize}
\tightlist

\item \textbf{CPU 2.0 GHz o superior}: Un procesador de 2 GHz es capaz de manejar las instrucciones por segundo necesarias para realizar las tareas requeridas en esta aplicación. Como el procesamiento de audio y la ejecución de la predicción.

\item \textbf{Memoria RAM de 4GB o superior}: 

\item \textbf{2GB de almacenamiento libre}: Esta cantidad de almacenamiento libre es la mínima necesaria para instalar la aplicación y sus librerías necesarias.

\item \textbf{Sistema operativo compatible con Python 3.10 o superior}: La aplicación está escrita en Python, por lo que se requiere un sistema operativo que pueda soportar la versión de Python 3.10 o superior.

\end{itemize}

\section{Instalación y ejecución}

\subsection{Instalación local}

Para obtener el código fuente del proyecto los pasos a seguir son los siguientes:

\begin{itemize}
\tightlist

\item \textbf{Clonar el repositorio Git}
	\begin{itemize}
	\tightlist
		\item \texttt{git clone https://github.com/jle1001/AI-Music.git}
	\end{itemize}

\item \textbf{Crear un entorno virtual de Python}: 
	\begin{itemize}
	\tightlist
		\item \texttt{python3 -m venv venv}
	\end{itemize}

\item \textbf{Activar el entorno virtual}: 
	\begin{itemize}
		\item Linux: \texttt{source venv/bin/activate}
		
		\item Windows: \texttt{venv\textbackslash Scripts\textbackslash activate}
	\end{itemize}

\item \textbf{Instalación de dependencias}
	\begin{itemize}
	\tightlist
		\item \texttt{pip install -r requirements.txt}
	\end{itemize}

\item \textbf{Iniciar el servidor}
	\begin{itemize}
	\tightlist
		\item \texttt{flask run --debug}
	\end{itemize}
\end{itemize}

Una vez que el servidor está en funcionamiento, se puede acceder a la aplicación web a través de un navegador entrando a la dirección \texttt{localhost:5000}.

\subsection{Uso en línea}

Para utilizar la aplicación en línea hay que entrar al siguiente enlace: 

\section{Manual del usuario}
El objetivo de este manual es proporcionar una guía paso a paso para que el usuario pueda utilizar las funciones de la aplicación.

\subsection{Subida de canciones}
\begin{itemize}
\tightlist

\item En la ventana principal de la aplicación hay que pulsar el botón \textit{Upload File}.
\item Elegir el fichero de audio en el almacenamiento.
\item Una vez subido el fichero de audio se darán opciones como: reproducir la canción, visualizar datos básicos y elegir el modelo para la predicción.

\end{itemize}

INSERTAR CAPTURA DE PANTALLA

\subsection{Análisis y predicción}

\begin{itemize}
\tightlist

\item En la ventana principal de la aplicación, pulsar el botón \textit{Upload File}.
\item Elegir el fichero de audio en el almacenamiento.
\item Una vez subido el fichero de audio se darán opciones como: reproducir la canción, visualizar datos básicos y elegir el modelo para la predicción.
\item Para realizar la predicción del género musical se debe elegir un modelo a utilizar y pulsar el botón \textit{Predict}.

\end{itemize}

INSERTAR CAPTURA DE PANTALLA

\subsection{Visualización del resultado}

\begin{itemize}
\tightlist

\item En la ventana principal de la aplicación, pulsar el botón \textit{Upload File}.
\item Elegir el fichero de audio en el almacenamiento.
\item Una vez subido el fichero de audio se darán opciones como: reproducir la canción, visualizar datos básicos y elegir el modelo para la predicción.
\item Para realizar la predicción del género musical se debe elegir un modelo a utilizar y pulsar el botón \textit{Predict}.
\item Una vez el modelo haya terminado de predecir el estilo musical, este se mostrará en la parte superior de la pantalla.
\item En la parte central de la pantalla se podrán ver cuatro gráficos con diferentes características musicales.

\end{itemize}

INSERTAR CAPTURA DE PANTALLA