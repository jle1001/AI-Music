\capitulo{2}{Objetivos del proyecto}
El objetivo general de este proyecto es diseñar y desarrollar un sistema de reconocimiento automático de estilos musicales mediante inteligencia artificial.

Para alcanzar este fin, se plantean los siguientes objetivos:

\subsection{Recogida de datos}
El proceso de recogida de datos consiste en recopilar y organizar diversos tipos de información relacionada con los datos que queremos estudiar. En este proyecto, se incluyen las pistas de audio y los metadatos obtenidos del conjunto de datos FMA (Free Music Archive).
\subsection{Preprocesamiento de datos}
El preprocesamiento de datos implica una serie de pasos destinados a limpiar, transformar y preparar los datos para asegurarque presentan un formato adecuado.
\subsection{Procesamiento de datos}
El procesamiento de datos en este proyecto significa extraer características relevantes para obtener información para realizar la clasificación. Características como MFCCs o espectogramas son adecuadas para esta tarea.
\subsection{Selección e implementación de modelos}
La selección e implementación de modelos implica la elección de los mejores algoritmos adecuados para la tarea.
\subsection{Evaluación de modelos}
La evaluación de modelos implica medir el rendimiento de los modelos de aprendizaje automático implementados utilizando métricas como la tasa de acierto, precisión o F1-Score.
\subsection{Interfaz de usuario}
El desarrollo de una interfaz de usuario permite al usuario interactuar con el modelo de reconocimiento musical de forma sencilla.
\subsection{Despliegue}
Desplegar la aplicación web en un servidor web o en una plataforma en la nube para garantizar que es accesible a los usuarios.
\subsection{Pruebas y control de calidad}
Probar la aplicación y la interfaz web para identificar y corregir problemas. Este proceso es crucial en todo el ciclo de vida de la aplicación.