\capitulo{2}{Objetivos del proyecto}
\section{Objetivos principales}
\begin{itemize}
\tightlist
\item Desarrollar un sistema de reconocimiento de estilos musicales utilizando inteligencia artificial.
\item Diseñar e implementar una aplicación web que permita usar el modelo de una forma sencilla.
\item Obtener conclusiones y conocimiento a partir de los datos.
\end{itemize}

\section{Objetivos técnicos}
\begin{itemize}
\tightlist
\item Desarrollar un sistema de \textit{aprendizaje automático} que, utilizando pistas de audio, detecte de forma automática el estilo musical basándose en diversas características.
\item Evaluar el modelo con diversas métricas como \textit{accuracy}, \textit{precision} o \textit{F1}.
\item Implementar tests unitarios y de integración.
\item Desarrollar un aplicación web utilizando librerías como Flask.
\item Desplegar la aplicación con el modelo entrenado a una plataforma online como Heroku.
\item Utilizar un sistema de control de versiones como Git para poder seguir los cambios en el código.
\item Seguir una metodología ágil como SCRUM para la gestión del desarrollo del proyecto.
\end{itemize}

\section{Objetivos personales}
\begin{itemize}
\tightlist
\item Adentrarse en el aprendizaje de las técnicas de deep learning y en las bibliotecas que dan soporte a este tipo de algoritmos.
\item Profundizar en el desarrollo web y APIs.
\item Aprender y obtener conclusiones mediante la lectura de artículos científicos.
\end{itemize}