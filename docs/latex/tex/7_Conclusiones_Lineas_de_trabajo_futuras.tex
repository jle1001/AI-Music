\capitulo{7}{Conclusiones y Líneas de trabajo futuras}

\section{Conclusiones}
A través de este proyecto, se ha demostrado la eficacia pero también la complejidad del uso de la Inteligencia Artificial (IA) para la clasificación automática de estilos musicales. 

En líneas generales, el sistema ha demostrado tener una eficacia razonable en la tarea de clasificar estilos musicales. No obstante, en el proceso también se han identificado limitaciones.

En primer lugar, aunque el proceso de extracción de características de audio juega un papel crucial en el desarrollo y entrenamiento de los modelos de clasificación de estilos musicales, podría ser conveniente considerar la exploración de otros métodos o la implementación de técnicas más sofisticadas con el objetivo de enriquecer la calidad de los datos que alimentan a la red neuronal.

Podría ser beneficioso explorar técnicas de \textit{deep learning} como el aprendizaje por transferencia (\textit{transfer learning}), los modelos \textit{pre-entrenados} podrían usarse como un punto de partida, lo que podría mejorar la eficiencia del aprendizaje y potencialmente mejorar la precisión del modelo.

La cantidad del conjunto de datos tiene un impacto directo en la eficacia del sistema. Utilizar un conjunto de datos más extenso y variado podría mejorar la capacidad del sistema para generalizar.

\textbf{En conclusión}, este proyecto ha servido para demostrar que la clasificación de estilos musicales utilizando IA es factible y que la inteligencia artificial aplicada a la música ofrece un gran potencial y motivación para investigaciones futuras.

\section{Líneas de trabajo futuras}

\begin{enumerate}
\item \textbf{Experimentación con diferentes modelos}: Se han utilizado principalmente redes neuronales convolucionales en este proyecto, pero hay muchas otras arquitecturas de red. Por ejemplo, las redes neuronales recurrentes (RNN) o modelos generativos adversarios (GANs) podrían ser buenas opciones para experimentar.

\item \textbf{Explorar enfoques de aprendizaje semi-supervisado y no supervisado}: Dada la dificultad de etiquetar grandes conjuntos de datos musicales, también se podría considerar la utilización de técnicas de aprendizaje semi-supervisado o no supervisado. Estos enfoques podrían ser útiles para aprovechar datos no etiquetados y mejorar el rendimiento del sistema.

\item \textbf{Incorporación de feedback del usuario}: Permitir a los usuarios corregir las clasificaciones incorrectas del modelo y utilizar esa información para mejorar el modelo.

\item \textbf{Construcción de un sistema de recomendación}: Teniendo el modelo de clasificación de estilos musicales, podría ser interesante construir un sistema de recomendación musical que sugiera canciones a los usuarios basándose en sus preferencias y hábitos de escucha.
\end{enumerate}