\capitulo{6}{Trabajos relacionados}

\section{Estado del arte}

El estado del arte es una expresión utilizada para describir el estado más alto de desarrollo en un cierto campo en el momento actual. 
En este apartado se va a investigar y exponer cuál es el método o el estudio más avanzado y con mayor acierto a la hora de clasificar géneros musicales.

\subsection{Estudios recientes}



\section{Trabajos relacionados}

\begin{itemize}
\tightlist

\item \textbf{Zero-shot Learning for Audio-based Music Classification and Tagging} \cite{choi2020zeroshot}: El artículo explora el uso del \textit{zero-shot learning} en la clasificación musical. 
Se aborda el problema de los datos sin etiquetar utilizando un espacio semántico adicional de etiquetas para descubrir la relación entre ellas. 
Los resultados de este estudio son prometedores y genera conclusiones y líneas futuras de trabajo como el usar las letras de las canciones para obtener información relevante sin intervención humana.

\item \textbf{Detecting Music Genre Using Extreme Gradient Boosting} \cite{10.1145/3184558.3191822}: Estudio en el que se utilizan diversos métodos de clasificación para predecir géneros musicales. Se trata de un estudio muy completo ya
que utiliza diversos métodos como CNN (\textit{convolutional neural network}), DNN (\textit{deep neural network}), árboles de decisión y ensembles (\textit{SGBoost, ExtraTrees}).

\item \textbf{Environmental sound classification using temporal-frequency attention based convolutional neural network} \cite{Mu_Yin_Huang_Xu_Du_2021}: Artículo que explora el uso de TFCNN (temporal-frequency attention based convolutional neural network model)
para clasificar sonidos ambientales. Es interesante observar los resultados, con una tasa de acierto superior al 90\% en casi todos los escenarios.
\end{itemize}
