\capitulo{1}{Introducción}

\section{Contexto}

La música es un importante elemento cultural con una gran variedad de estilos y géneros.
La identificación automática de estilos musicales es un campo de estudio emergente, cuyo objetivo es crear sistemas inteligentes capaces de reconocer y clasificar automáticamente el estilo musical de una canción o pieza musical sin intervención humana. 

Este ámbito de estudio tiene importantes aplicaciones en áreas como la recomendación musical o la organización del audio digital.

La aplicación de algoritmos de inteligencia artificial ha mostrado un éxito considerable en este ámbito.

El presente trabajo tiene como objetivo desarrollar un sistema de reconocimiento automático de estilos musicales utilizando procesamiento de audio y algoritmos de inteligencia artificial.

\section{Estructura del proyecto}
El presente documento contiene la siguiente estructura.

\begin{itemize}
\tightlist
\item \textbf{Introducción}: contexto, descripción del problema, estructura del proyecto y materiales relevantes.
\item \textbf{Objetivos del proyecto}: explicación y exposición de los objetivos del proyecto, tanto generales como técnicos.
\item \textbf{Conceptos teóricos}: breve explicación de los conceptos teóricos principales que son necesarios para entender el problema a resolver.
\item \textbf{Técnicas y herramientas}: presentación de la metodología, técnicas y herramientas utilizadas en el desarrollo del proyecto.
\item \textbf{Aspectos relevantes del desarrollo del proyecto}: aspectos importantes en el desarrollo del proyecto.
\item \textbf{Trabajos relacionados}: estado del arte y trabajos relacionados en el área de estudio.
\item \textbf{Conclusiones y líneas de trabajo futuras}: conclusiones obtenidas tras la finalización del proyecto y posibles líneas de trabajo futuras.
\end{itemize}

Además, el documento \texttt{anexos} contiene la siguiente estructura:

\begin{itemize}
\tightlist
\item \textbf{Apéndice A. Plan de Proyecto Software}:
\item \textbf{Apéndice B. Especificación de requisitos}:
\item \textbf{Apéndice C. Especificación de diseño}:
\item \textbf{Apéndice D. Documentación técnica de programación}:
\item \textbf{Apéndice E. Documentación de usuario}:
\end{itemize}

\section{Materiales}

\begin{itemize}
\tightlist
\item \textbf{Repositorio}: \url{https://github.com/jle1001/AI-Music}
\item \textbf{Aplicación}:
\end{itemize}