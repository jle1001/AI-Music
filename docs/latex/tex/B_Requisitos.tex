\apendice{Especificación de Requisitos}

\section{Introducción}

\section{Objetivos generales}

\begin{itemize}
\tightlist
\item Desarrollar un sistema de reconocimiento de estilos musicales utilizando inteligencia artificial.
\item Diseñar e implementar una aplicación web que permita usar el modelo de una forma sencilla.
\item Obtener conclusiones y conocimiento a partir de los datos.
\end{itemize}

\section{Catalogo de requisitos}

\subsection{Requisitos funcionales}

Un requisito funcional es una especificación que describe lo que un sistema debe hacer o cómo debe comportarse.
Los requisitos funcinales pueden incluir detalles como cálculos, manipulación de datos, interacción con el usuario, etc.
A continuación se listan los requisitos funcionales extraidos para este proyecto:

RF-1: Predicción de estilo musical.

RF-2: Presentación de la información.

RF-3: Implementación de una interfaz de usuario.

\subsection{Requisitos no funcionales}

Los requisitos no funcionales se centran en las características del sistema que no están directamente relacionadas con su comportamiento, por ejemplo rendimiento o seguridad.
A continuación se listan los requisistos no funcionales extraídos para este proyecto:

RNF-1: Rendimiento

RNF-2: Seguridad

RNF-3: Compatibilidad

RNF-4: Usabilidad

RNF-5: Portabilidad

RNF-6: Adaptabilidad

\section{Especificación de requisitos}

Los casos de uso son descripciones detalladas del funcionamiento de una parte del sistema desde la perspectiva del usuario.

En este sección se mostrará el diagrama general de casos de uso y se explicará en detalle cada uno de ellos.

\subsection{Diagrama general de casos de uso}

PONER DIAGRAMA GENERAL DE CASOS DE USO

\subsection{Casos de uso}

% Caso de Uso 1 -> Consultar Experimentos.
\begin{table}[p]
	\centering
	\begin{tabularx}{\linewidth}{ p{0.21\columnwidth} p{0.71\columnwidth} }
		\toprule
		\textbf{CU-1}    & \textbf{Ejemplo de caso de uso}\\
		\toprule
		\textbf{Versión}              & 1.0    \\
		\textbf{Autor}                & José Ángel López \\
		\textbf{Requisitos asociados} & RF-xx, RF-xx \\
		\textbf{Descripción}          & La descripción del CU \\
		\textbf{Precondición}         & Precondiciones (podría haber más de una) \\
		\textbf{Acciones}             &
		\begin{enumerate}
			\def\labelenumi{\arabic{enumi}.}
			\tightlist
			\item Pasos del CU
			\item Pasos del CU (añadir tantos como sean necesarios)
		\end{enumerate}\\
		\textbf{Postcondición}        & Postcondiciones (podría haber más de una) \\
		\textbf{Excepciones}          & Excepciones \\
		\textbf{Importancia}          & Alta o Media o Baja... \\
		\bottomrule
	\end{tabularx}
	\caption{CU-1 Nombre del caso de uso.}
\end{table}