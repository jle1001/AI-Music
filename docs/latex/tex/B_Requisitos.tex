\apendice{Especificación de Requisitos}

\section{Introducción}

Esta sección describe la especificación de requisitos para el proyecto. Se proporciona información detallada sobre los requisitos funcionales (RF) y requisitos no funcionales (RNF) así como el detalle de los casos de uso (CU).

\section{Objetivos generales}

\begin{itemize}
\tightlist
\item Desarrollar un sistema de reconocimiento de estilos musicales utilizando inteligencia artificial.
\item Diseñar e implementar una aplicación web que permita usar el modelo de una forma sencilla.
\item Obtener conclusiones y conocimiento a partir de los datos.
\end{itemize}

\section{Catalogo de requisitos}

\subsection{Requisitos funcionales}

Un requisito funcional es una especificación que describe lo que un sistema debe hacer o cómo debe comportarse.
Los requisitos funcionales pueden incluir detalles como cálculos, manipulación de datos, interacción con el usuario, etc.
A continuación se listan los requisitos funcionales extraidos para este proyecto:

\begin{itemize}
\tightlist

\item \textbf{RF-1 Predicción de estilo musical}: la aplicación debe ser capaz de predecir el estilo musical de una canción.
	\begin{itemize}
	\tightlist
	\item \textbf{RF-1.1 Extracción de características de audio}: la aplicación debe ser capaz de extraer características relevantes de la canción que se utilizarán para predecir el estilo musical.

	\item \textbf{RF-1.2 Implementar un modelo de machine learning para realizar la predicción}: la aplicación debe implementar un modelo de machine learning que se entrenará con las características de un conjunto de datos musical.
	\end{itemize}

\item \textbf{RF-2 Presentación de la información}: la aplicación debe mostrar información relevante sobre la predicción realizada.
	\begin{itemize}
	\tightlist
	\item \textbf{RF-2.1 Detalles de la predicción}: la aplicación debe ser capaz de mostrar datos sobre los detalles de la predicción como la confianza (probabilidad) o posibles géneros musicales alternativos.

	\item \textbf{RF-2.2 Información de la canción}: la aplicación debe mostrar información relevante sobre la pista de audio o canción utilizada como la duración, formato o frecuencia de muestreo.
	\end{itemize}
\item \textbf{RF-3 Implementación de una interfaz de usuario}: la aplicación debe contener una interfaz de usuario para facilitar su uso.
	\begin{itemize}
	\tightlist
	\item \textbf{RF-3.1 Funcionalidad de carga de archivos}: la interfaz de usuario debe incluir una función de carga de archivos que permita seleccionar y cargar una canción para realizar la predicción.

	\item \textbf{RF-3.2 Representación de los resultados}: la interfaz de usuario debe incluir sección donde se presenten de forma clara los resultados de la predicción.
	\end{itemize}

\end{itemize}
\subsection{Requisitos no funcionales}

Los requisitos no funcionales se centran en las características del sistema que no están directamente relacionadas con su comportamiento, por ejemplo rendimiento o seguridad.
A continuación se listan los requisistos no funcionales extraídos para este proyecto:

\begin{itemize}
\tightlist
\item \textbf{RNF-1 Rendimiento}: la aplicación debe ser capaz de procesar y analizar una canción y devolver los resultados dentro de un tiempo razonable, idealmente unos pocos segundos.

\item \textbf{RNF-2 Seguridad}: la aplicación debe cumplir con protocolos de seguridad como cifrado de comunicaciones o la eliminación de la canción después de su procesamiento.

\item \textbf{RNF-3 Compatibilidad}: la aplicación debe ser compatible con diversos navegadores y sistemas operativos.

\item \textbf{RNF-4 Usabilidad}: la aplicación debe ser fácil de usar para el usuario. El usuario debería ser capaz de realizar el proceso de subida de archivos y comprobar la predicción y resultados de una forma sencilla.

\item \textbf{RNF-5 Portabilidad}: la aplicación debe ser compatible con diversos dispositivos como PCs, tablets o dispositivos móviles.
\end{itemize}

\section{Especificación de requisitos}

Los casos de uso son descripciones detalladas del funcionamiento de una parte del sistema desde la perspectiva del usuario.

En este sección se mostrará el diagrama general de casos de uso y se explicará en detalle cada uno de ellos.

\subsection{Diagrama general de casos de uso}

\imagen{Diagrama_general_de_casos_de_uso}{Diagrama general de casos de uso.}

\subsection{Casos de uso}

A continuación se exponen las especificaciones de los casos de uso que forman la aplicación.

% Caso de Uso 1 -> Carga de la canción
\begin{table}[p]
	\centering
	\begin{tabularx}{\linewidth}{ p{0.21\columnwidth} p{0.71\columnwidth} }
		\toprule
		\textbf{CU-1}    & \textbf{Carga de la canción}\\
		\toprule
		\textbf{Versión}              & 1.0    \\
		\textbf{Autor}                & José Ángel López \\
		\textbf{Requisitos asociados} & RF-1.1, RF-3.1 \\
		\textbf{Descripción}          & El usuario utiliza la función de carga de archivo para seleccionar y cargar una canción en la aplicación. \\
		\textbf{Precondiciones}        & 
		\begin{enumerate}		    
			\def\labelenumi{\arabic{enumi}.}
			\tightlist
			\item La aplicación se encuentra en funcionamiento.
			\item El usuario tiene acceso a los archivos de música que desea cargar.
		\end{enumerate}\\
		\textbf{Acciones}             &
		\begin{enumerate}
			\def\labelenumi{\arabic{enumi}.}
			\tightlist
			\item El usuario selecciona la opción para cargar una canción pulsando el botón \textit{Upload File}.
			\item El usuario busca y selecciona la canción que desea cargar desde su almacenamiento local.
			\item La canción se carga en la aplicación.
		\end{enumerate}\\
		\textbf{Postcondiciones} &
		\begin{enumerate}
			\def\labelenumi{\arabic{enumi}.}
			\tightlist
			\item La canción se encuentra cargada en la aplicación.
			\item La aplicación está lista para extraer características y realizar el proceso de predicción de la canción cargada.
		\end{enumerate}\\
		\textbf{Excepciones} &
		\begin{enumerate}
			\def\labelenumi{\arabic{enumi}.}
			\tightlist
			\item El archivo cargado no es un fichero de audio.
			\item La canción tiene un formato incompatible.
		\end{enumerate}\\
		\textbf{Importancia}          & Alta \\
		\bottomrule
	\end{tabularx}
	\caption{CU-1 Carga de la canción}
\end{table}

% Caso de Uso 2 -> Extracción de características de audio.
\begin{table}[p]
	\centering
	\begin{tabularx}{\linewidth}{ p{0.21\columnwidth} p{0.71\columnwidth} }
		\toprule
		\textbf{CU-2}    & \textbf{Extracción de características de audio.}\\
		\toprule
		\textbf{Versión}              & 1.0    \\
		\textbf{Autor}                & José Ángel López \\
		\textbf{Requisitos asociados} & RF-1.1 \\
		\textbf{Descripción}          & La aplicación extrae las características relevantes del audio para su posterior análisis. \\
		\textbf{Precondiciones}        & 
		\begin{enumerate}		    
			\def\labelenumi{\arabic{enumi}.}
			\tightlist
			\item La canción se ha cargado correctamente en la aplicación.
		\end{enumerate}\\
		\textbf{Acciones}             &
		\begin{enumerate}
			\def\labelenumi{\arabic{enumi}.}
			\tightlist
			\item La aplicación inicia el proceso de extracción de características del archivo de audio cargado.
		\end{enumerate}\\
		\textbf{Postcondiciones} &
		\begin{enumerate}
			\def\labelenumi{\arabic{enumi}.}
			\tightlist
			\item Las características relevantes de la canción se han extraído correctamente.
			\item La aplicación está lista para utilizar las características extraídas para realizar la predicción del estilo musical.
		\end{enumerate}\\
		\textbf{Excepciones} &
		\begin{enumerate}
			\def\labelenumi{\arabic{enumi}.}
			\tightlist
			\item Ocurrió un error durante la extracción de las características del audio.
		\end{enumerate}\\
		\textbf{Importancia}          & Alta \\
		\bottomrule
	\end{tabularx}
	\caption{CU-2 Extracción de características de audio}
\end{table}

% Caso de Uso 3 -> Ejecución del modelo de machine learning
\begin{table}[p]
	\centering
	\begin{tabularx}{\linewidth}{ p{0.21\columnwidth} p{0.71\columnwidth} }
		\toprule
		\textbf{CU-3}    & \textbf{Ejecución del modelo de machine learning}\\
		\toprule
		\textbf{Versión}              & 1.0    \\
		\textbf{Autor}                & José Ángel López \\
		\textbf{Requisitos asociados} & RF-1.2 \\
		\textbf{Descripción}          & La aplicación ejecuta el modelo de machine learning para realizar la predicción del estilo musical según las caracterísiticas extraídas. \\
		\textbf{Precondiciones}        & 
		\begin{enumerate}		    
			\def\labelenumi{\arabic{enumi}.}
			\tightlist
			\item Las características de audio de la canción se han extraído correctamente.
			\item Las características de audio de la canción son suficientes para realizar el proceso de predicción.
		\end{enumerate}\\
		\textbf{Acciones}             &
		\begin{enumerate}
			\def\labelenumi{\arabic{enumi}.}
			\tightlist
			\item El modelo procesa las características extraídas de la canción.
			\item Se obtiene el resultado de la predicción, el cual consiste en un vector de probabilidades entre diferentes estilos musicales.
		\end{enumerate}\\
		\textbf{Postcondiciones} &
		\begin{enumerate}
			\def\labelenumi{\arabic{enumi}.}
			\tightlist
			\item La aplicación ha generado una predicción sobre el estilo musical de la canción.
			\item La aplicación está lista para mostrar los detalles de la predicción al usuario.
		\end{enumerate}\\
		\textbf{Excepciones} &
		\begin{enumerate}
			\def\labelenumi{\arabic{enumi}.}
			\tightlist
			\item El modelo no puede procesar las características extraídas.
		\end{enumerate}\\
		\textbf{Importancia}          & Alta \\
		\bottomrule
	\end{tabularx}
	\caption{CU-3 Ejecución del modelo de machine learning}
\end{table}

% Caso de Uso 4 -> Visualización de la predicción
\begin{table}[p]
	\centering
	\begin{tabularx}{\linewidth}{ p{0.21\columnwidth} p{0.71\columnwidth} }
		\toprule
		\textbf{CU-4}    & \textbf{Visualización de la predicción}\\
		\toprule
		\textbf{Versión}              & 1.0    \\
		\textbf{Autor}                & José Ángel López \\
		\textbf{Requisitos asociados} & RF-2.1, RF-2.2 \\
		\textbf{Descripción}          & Tras la ejecución del modelo de machine learning, la aplicación muestra la predicción del estilo musical al usuario. \\
		\textbf{Precondiciones}        & 
		\begin{enumerate}		    
			\def\labelenumi{\arabic{enumi}.}
			\tightlist
			\item La aplicación ha generado una predicción.
		\end{enumerate}\\
		\textbf{Acciones}             &
		\begin{enumerate}
			\def\labelenumi{\arabic{enumi}.}
			\tightlist
			\item La aplicación muestra la predicción del estilo musical en la interfaz.
			\item El usuario puede cambiar entre distintas visualizaciones de forma dinámica.
			\item El usuario visualiza la predicción.
		\end{enumerate}\\
		\textbf{Postcondiciones} &
		\begin{enumerate}
			\def\labelenumi{\arabic{enumi}.}
			\tightlist
			\item El usuario ha obtenido la predicción del estilo musical de la canción.
		\end{enumerate}\\
		\textbf{Excepciones} &
		\begin{enumerate}
			\def\labelenumi{\arabic{enumi}.}
			\tightlist
			\item Se produce un error al intentar mostrar la predicción.
		\end{enumerate}\\
		\textbf{Importancia}          & Alta \\
		\bottomrule
	\end{tabularx}
	\caption{CU-4 Visualización de la predicción}
\end{table}

% Caso de Uso 5 -> Reproducción de la canción
\begin{table}[p]
	\centering
	\begin{tabularx}{\linewidth}{ p{0.21\columnwidth} p{0.71\columnwidth} }
		\toprule
		\textbf{CU-5}    & \textbf{Reproducción de la canción}\\
		\toprule
		\textbf{Versión}              & 1.0    \\
		\textbf{Autor}                & José Ángel López \\
		\textbf{Requisitos asociados} & RF-1.2, RF-1.3 \\
		\textbf{Descripción}          & La aplicación permite al usuario reproducir la canción cargada. \\
		\textbf{Precondiciones}        & 
		\begin{enumerate}		    
			\def\labelenumi{\arabic{enumi}.}
			\tightlist
			\item La canción ha sido cargada correctamente en la aplicación.
		\end{enumerate}\\
		\textbf{Acciones}             &
		\begin{enumerate}
			\def\labelenumi{\arabic{enumi}.}
			\tightlist
			\item El usuario selecciona la opción de reproducir la canción mediante el botón \textit{play} del reproductor.
			\item La aplicación reproduce la canción cargada.
		\end{enumerate}\\
		\textbf{Postcondiciones} &
		\begin{enumerate}
			\def\labelenumi{\arabic{enumi}.}
			\tightlist
			\item La canción está siendo reproducida en la aplicación.
		\end{enumerate}\\
		\textbf{Excepciones} &
		\begin{enumerate}
			\def\labelenumi{\arabic{enumi}.}
			\tightlist
			\item Se produce un error en la reproducción de la canción.
		\end{enumerate}\\
		\textbf{Importancia}          & Media \\
		\bottomrule
	\end{tabularx}
	\caption{CU-5 Reproducción de la canción}
\end{table}

% Caso de Uso 6 -> Carga de una nueva canción
\begin{table}[p]
	\centering
	\begin{tabularx}{\linewidth}{ p{0.21\columnwidth} p{0.71\columnwidth} }
		\toprule
		\textbf{CU-6}    & \textbf{Carga de una nueva canción}\\
		\toprule
		\textbf{Versión}              & 1.0    \\
		\textbf{Autor}                & José Ángel López \\
		\textbf{Requisitos asociados} & RF-1.1, RF-1.3 \\
		\textbf{Descripción}          & El usuario puede cargar una nueva canción sin tener que cerrar o reiniciar la aplicación. \\
		\textbf{Precondiciones}        & 
		\begin{enumerate}		    
			\def\labelenumi{\arabic{enumi}.}
			\tightlist
			\item La aplicación ha realizado la predicción de una canción.
		\end{enumerate}\\
		\textbf{Acciones}             &
		\begin{enumerate}
			\def\labelenumi{\arabic{enumi}.}
			\tightlist
			\item El usuario hace selecciona el nombre de la aplicación, en la zona superior izquierda de la pantalla, para volver a la página inicial.
			\item El usuario pulsa el botón \textit{Upload File}.
			\item El usuario busca y selecciona la nueva canción que desea cargar desde su almacenamiento.
		\end{enumerate}\\
		\textbf{Postcondiciones} &
		\begin{enumerate}
			\def\labelenumi{\arabic{enumi}.}
			\tightlist
			\item La nueva canción se carga en la aplicación.
		\end{enumerate}\\
		\textbf{Excepciones} &
		\begin{enumerate}
			\def\labelenumi{\arabic{enumi}.}
			\tightlist
			\item El archivo cargado no es un fichero de audio.
			\item La canción tiene un formato incompatible.
		\end{enumerate}\\
		\textbf{Importancia}          & Alta \\
		\bottomrule
	\end{tabularx}
	\caption{CU-6 Carga de una nueva canción}
\end{table}