\apendice{Especificación de Requisitos}

\section{Introducción}

\section{Objetivos generales}

\begin{itemize}
\tightlist
\item Desarrollar un sistema de reconocimiento de estilos musicales utilizando inteligencia artificial.
\item Diseñar e implementar una aplicación web que permita usar el modelo de una forma sencilla.
\item Obtener conclusiones y conocimiento a partir de los datos.
\end{itemize}

\section{Catalogo de requisitos}

ENUMERAR UNO A UNO LOS REQUISITOS FUNCIONALES Y NO FUNCIONALES DE LA APLICACIÓN

\section{Especificación de requisitos}

MOSTRAR DIAGRAMA DE CASOS DE USO, Y DESARROLLO DE CADA REQUISITO

% Caso de Uso 1 -> Consultar Experimentos.
\begin{table}[p]
	\centering
	\begin{tabularx}{\linewidth}{ p{0.21\columnwidth} p{0.71\columnwidth} }
		\toprule
		\textbf{CU-1}    & \textbf{Ejemplo de caso de uso}\\
		\toprule
		\textbf{Versión}              & 1.0    \\
		\textbf{Autor}                & José Ángel López \\
		\textbf{Requisitos asociados} & RF-xx, RF-xx \\
		\textbf{Descripción}          & La descripción del CU \\
		\textbf{Precondición}         & Precondiciones (podría haber más de una) \\
		\textbf{Acciones}             &
		\begin{enumerate}
			\def\labelenumi{\arabic{enumi}.}
			\tightlist
			\item Pasos del CU
			\item Pasos del CU (añadir tantos como sean necesarios)
		\end{enumerate}\\
		\textbf{Postcondición}        & Postcondiciones (podría haber más de una) \\
		\textbf{Excepciones}          & Excepciones \\
		\textbf{Importancia}          & Alta o Media o Baja... \\
		\bottomrule
	\end{tabularx}
	\caption{CU-1 Nombre del caso de uso.}
\end{table}