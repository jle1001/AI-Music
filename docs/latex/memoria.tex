\documentclass[a4paper,12pt,twoside]{memoir}

% Castellano
\usepackage[spanish,es-tabla]{babel}
\selectlanguage{spanish}
\usepackage[utf8]{inputenc}
\usepackage[T1]{fontenc}
\usepackage{lmodern} % Scalable font
\usepackage{microtype}
\usepackage{placeins}

\RequirePackage{booktabs}
\RequirePackage[table]{xcolor}
\RequirePackage{xtab}
\RequirePackage{multirow}

% Links
\PassOptionsToPackage{hyphens}{url}\usepackage[colorlinks]{hyperref}
\hypersetup{
	allcolors = {red}
}

% Ecuaciones
\usepackage{amsmath}

% Gráficos
\usepackage{tikz}

% Rutas de fichero / paquete
\newcommand{\ruta}[1]{{\sffamily #1}}

% Párrafos
\nonzeroparskip

% Huérfanas y viudas
\widowpenalty100000
\clubpenalty100000

% Imágenes

% Comando para insertar una imagen en un lugar concreto.
% Los parámetros son:
% 1 --> Ruta absoluta/relativa de la figura
% 2 --> Texto a pie de figura
% 3 --> Tamaño en tanto por uno relativo al ancho de página
\usepackage{graphicx}
\newcommand{\imagen}[3]{
	\begin{figure}[!h]
		\centering
		\includegraphics[width=#3\textwidth]{#1}
		\caption{#2}\label{fig:#1}
	\end{figure}
	\FloatBarrier
}

% Comando para insertar una imagen sin posición.
% Los parámetros son:
% 1 --> Ruta absoluta/relativa de la figura
% 2 --> Texto a pie de figura
% 3 --> Tamaño en tanto por uno relativo al ancho de página
\newcommand{\imagenflotante}[3]{
	\begin{figure}
		\centering
		\includegraphics[width=#3\textwidth]{#1}
		\caption{#2}\label{fig:#1}
	\end{figure}
}

% El comando \figura nos permite insertar figuras comodamente, y utilizando
% siempre el mismo formato. Los parametros son:
% 1 --> Porcentaje del ancho de página que ocupará la figura (de 0 a 1)
% 2 --> Fichero de la imagen
% 3 --> Texto a pie de imagen
% 4 --> Etiqueta (label) para referencias
% 5 --> Opciones que queramos pasarle al \includegraphics
% 6 --> Opciones de posicionamiento a pasarle a \begin{figure}
\newcommand{\figuraConPosicion}[6]{%
  \setlength{\anchoFloat}{#1\textwidth}%
  \addtolength{\anchoFloat}{-4\fboxsep}%
  \setlength{\anchoFigura}{\anchoFloat}%
  \begin{figure}[#6]
    \begin{center}%
      \Ovalbox{%
        \begin{minipage}{\anchoFloat}%
          \begin{center}%
            \includegraphics[width=\anchoFigura,#5]{#2}%
            \caption{#3}%
            \label{#4}%
          \end{center}%
        \end{minipage}
      }%
    \end{center}%
  \end{figure}%
}

%
% Comando para incluir imágenes en formato apaisado (sin marco).
\newcommand{\figuraApaisadaSinMarco}[5]{%
  \begin{figure}%
    \begin{center}%
    \includegraphics[angle=90,height=#1\textheight,#5]{#2}%
    \caption{#3}%
    \label{#4}%
    \end{center}%
  \end{figure}%
}
% Para las tablas
\newcommand{\otoprule}{\midrule [\heavyrulewidth]}
%
% Nuevo comando para tablas pequeñas (menos de una página).
\newcommand{\tablaSmall}[5]{%
 \begin{table}
  \begin{center}
   \rowcolors {2}{gray!35}{}
   \begin{tabular}{#2}
    \toprule
    #4
    \otoprule
    #5
    \bottomrule
   \end{tabular}
   \caption{#1}
   \label{tabla:#3}
  \end{center}
 \end{table}
}

%
% Nuevo comando para tablas pequeñas (menos de una página).
\newcommand{\tablaSmallSinColores}[5]{%
 \begin{table}[H]
  \begin{center}
   \begin{tabular}{#2}
    \toprule
    #4
    \otoprule
    #5
    \bottomrule
   \end{tabular}
   \caption{#1}
   \label{tabla:#3}
  \end{center}
 \end{table}
}

\newcommand{\tablaApaisadaSmall}[5]{%
\begin{landscape}
  \begin{table}
   \begin{center}
    \rowcolors {2}{gray!35}{}
    \begin{tabular}{#2}
     \toprule
     #4
     \otoprule
     #5
     \bottomrule
    \end{tabular}
    \caption{#1}
    \label{tabla:#3}
   \end{center}
  \end{table}
\end{landscape}
}

%
% Nuevo comando para tablas grandes con cabecera y filas alternas coloreadas en gris.
\newcommand{\tabla}[6]{%
  \begin{center}
    \tablefirsthead{
      \toprule
      #5
      \otoprule
    }
    \tablehead{
      \multicolumn{#3}{l}{\small\sl continúa desde la página anterior}\\
      \toprule
      #5
      \otoprule
    }
    \tabletail{
      \hline
      \multicolumn{#3}{r}{\small\sl continúa en la página siguiente}\\
    }
    \tablelasttail{
      \hline
    }
    \bottomcaption{#1}
    \rowcolors {2}{gray!35}{}
    \begin{xtabular}{#2}
      #6
      \bottomrule
    \end{xtabular}
    \label{tabla:#4}
  \end{center}
}

%
% Nuevo comando para tablas grandes con cabecera.
\newcommand{\tablaSinColores}[6]{%
  \begin{center}
    \tablefirsthead{
      \toprule
      #5
      \otoprule
    }
    \tablehead{
      \multicolumn{#3}{l}{\small\sl continúa desde la página anterior}\\
      \toprule
      #5
      \otoprule
    }
    \tabletail{
      \hline
      \multicolumn{#3}{r}{\small\sl continúa en la página siguiente}\\
    }
    \tablelasttail{
      \hline
    }
    \bottomcaption{#1}
    \begin{xtabular}{#2}
      #6
      \bottomrule
    \end{xtabular}
    \label{tabla:#4}
  \end{center}
}

%
% Nuevo comando para tablas grandes sin cabecera.
\newcommand{\tablaSinCabecera}[5]{%
  \begin{center}
    \tablefirsthead{
      \toprule
    }
    \tablehead{
      \multicolumn{#3}{l}{\small\sl continúa desde la página anterior}\\
      \hline
    }
    \tabletail{
      \hline
      \multicolumn{#3}{r}{\small\sl continúa en la página siguiente}\\
    }
    \tablelasttail{
      \hline
    }
    \bottomcaption{#1}
  \begin{xtabular}{#2}
    #5
   \bottomrule
  \end{xtabular}
  \label{tabla:#4}
  \end{center}
}



\definecolor{cgoLight}{HTML}{EEEEEE}
\definecolor{cgoExtralight}{HTML}{FFFFFF}

%
% Nuevo comando para tablas grandes sin cabecera.
\newcommand{\tablaSinCabeceraConBandas}[5]{%
  \begin{center}
    \tablefirsthead{
      \toprule
    }
    \tablehead{
      \multicolumn{#3}{l}{\small\sl continúa desde la página anterior}\\
      \hline
    }
    \tabletail{
      \hline
      \multicolumn{#3}{r}{\small\sl continúa en la página siguiente}\\
    }
    \tablelasttail{
      \hline
    }
    \bottomcaption{#1}
    \rowcolors[]{1}{cgoExtralight}{cgoLight}

  \begin{xtabular}{#2}
    #5
   \bottomrule
  \end{xtabular}
  \label{tabla:#4}
  \end{center}
}



\graphicspath{ {./img/} }

% Capítulos
\chapterstyle{bianchi}
\newcommand{\capitulo}[2]{
	\setcounter{chapter}{#1}
	\setcounter{section}{0}
	\setcounter{figure}{0}
	\setcounter{table}{0}
	\chapter*{#2}
	\addcontentsline{toc}{chapter}{#2}
	\markboth{#2}{#2}
}

% Apéndices
\renewcommand{\appendixname}{Apéndice}
\renewcommand*\cftappendixname{\appendixname}

\newcommand{\apendice}[1]{
	%\renewcommand{\thechapter}{A}
	\chapter{#1}
}

\renewcommand*\cftappendixname{\appendixname\ }

% Formato de portada
\makeatletter
\usepackage{xcolor}
\newcommand{\tutor}[1]{\def\@tutor{#1}}
\newcommand{\course}[1]{\def\@course{#1}}
\definecolor{cpardoBox}{HTML}{E6E6FF}
\def\maketitle{
  \null
  \thispagestyle{empty}
  % Cabecera ----------------
\noindent\includegraphics[width=\textwidth]{cabecera}\vspace{1cm}%
  \vfill
  % Título proyecto y escudo informática ----------------
  \colorbox{cpardoBox}{%
    \begin{minipage}{.8\textwidth}
      \vspace{.5cm}\Large
      \begin{center}
      \textbf{TFG del Grado en Ingeniería Informática}\vspace{.6cm}\\
      \textbf{\LARGE\@title{}}
      \end{center}
      \vspace{.2cm}
    \end{minipage}

  }%
  \hfill\begin{minipage}{.20\textwidth}
    \includegraphics[width=\textwidth]{escudoInfor}
  \end{minipage}
  \vfill
  % Datos de alumno, curso y tutores ------------------
  \begin{center}%
  {%
    \noindent\LARGE
    Presentado por \@author{}\\ 
    en Universidad de Burgos --- \@date{}\\
    Tutores: \@tutor{}\\
  }%
  \end{center}%
  \null
  \cleardoublepage
  }
\makeatother

\newcommand{\nombre}{José Ángel López Estrada} %%% cambio de comando

% Datos de portada
\title{AI-Music}
\author{\nombre}
\tutor{César Ignacio García Osorio, Alicia Olivares Gil}
\date{\today}

\begin{document}

\maketitle


\newpage\null\thispagestyle{empty}\newpage


%%%%%%%%%%%%%%%%%%%%%%%%%%%%%%%%%%%%%%%%%%%%%%%%%%%%%%%%%%%%%%%%%%%%%%%%%%%%%%%%%%%%%%%%
\thispagestyle{empty}


\noindent\includegraphics[width=\textwidth]{cabecera}\vspace{1cm}

\noindent D. nombre tutor, profesor del departamento de nombre departamento, área de nombre área.

\noindent Expone:

\noindent Que el alumno D. \nombre, con DNI 21071560H, ha realizado el Trabajo final de Grado en Ingeniería Informática titulado título de TFG. 

\noindent Y que dicho trabajo ha sido realizado por el alumno bajo la dirección del que suscribe, en virtud de lo cual se autoriza su presentación y defensa.

\begin{center} %\large
En Burgos, {\large \today}
\end{center}

\vfill\vfill\vfill

% Author and supervisor
\begin{minipage}{0.45\textwidth}
\begin{flushleft} %\large
Vº. Bº. del Tutor:\\[2cm]
D. nombre tutor
\end{flushleft}
\end{minipage}
\hfill
\begin{minipage}{0.45\textwidth}
\begin{flushleft} %\large
Vº. Bº. del co-tutor:\\[2cm]
D. nombre co-tutor
\end{flushleft}
\end{minipage}
\hfill

\vfill

% para casos con solo un tutor comentar lo anterior
% y descomentar lo siguiente
%Vº. Bº. del Tutor:\\[2cm]
%D. nombre tutor


\newpage\null\thispagestyle{empty}\newpage




\frontmatter

% Abstract en castellano
\renewcommand*\abstractname{Resumen}
\begin{abstract}
El rápido crecimiento del consumo de música digital ha generado una gran cantidad de datos de audio, lo que ha priorizado la necesidad de contar con métodos de clasificación y categorización. Plataformas como \textit{Spotify} o \textit{Apple Music} con bibliotecas que superan los 100 millones de canciones y un crecimiento constante de decenas de miles de canciones diariamente, resaltan la importancia de contar con una clasificación musical precisa y eficiente.

En este proyecto, se utilizan métodos de inteligencia artificial (IA) con el fin de clasificar de forma automática diferentes géneros musicales a partir de un análisis de las características del audio. 

La aplicación desarrollada en el proyecto se presenta como una plataforma web que permite a los usuarios cargar y analizar sus canciones.
\end{abstract}

\renewcommand*\abstractname{Descriptores}
\begin{abstract}
machine learning, big data, inteligencia artificial, clasificación, música, python, flask, servidor web \ldots
\end{abstract}

\clearpage

% Abstract en inglés
\renewcommand*\abstractname{Abstract}
\begin{abstract}
The rapid growth of digital music consumption has generated a large amount of audio data, which has prioritized the need for classification and categorization methods. Platforms like \textit{Spotify} or \textit{Apple Music}, with libraries exceeding 100 million songs and a constant growth of tens of thousands of songs daily, highlight the importance of having accurate and efficient music classification.

In this project, artificial intelligence (AI) methods are used to automatically classify different music genres based on an analysis of audio features.

The application developed in the project is presented as a web platform that allows users to upload and analyze their songs.
\end{abstract}

\renewcommand*\abstractname{Keywords}
\begin{abstract}
machine learning, big data, artificial intelligence, classification, music, python, flask, web server
\end{abstract}

\clearpage

% Indices
\tableofcontents

\clearpage

\listoffigures

\clearpage

\listoftables
\clearpage

\mainmatter
\capitulo{1}{Introducción}

\section{Contexto}

La música es un importante elemento cultural con una gran variedad de estilos y géneros.
La identificación automática de estilos musicales es un campo de estudio emergente, cuyo objetivo es crear sistemas inteligentes capaces de reconocer y clasificar automáticamente el estilo musical de una canción o pieza musical sin intervención humana. 

Este ámbito de estudio tiene importantes aplicaciones en áreas como la recomendación musical o la organización del audio digital.

La aplicación de algoritmos de inteligencia artificial ha mostrado un éxito considerable en este ámbito.

El presente trabajo tiene como objetivo desarrollar un sistema de reconocimiento automático de estilos musicales utilizando procesamiento de audio y algoritmos de inteligencia artificial.

\section{Estructura del proyecto}

TODO: ADD HERE PROJECT SECTIONS AND LINKS TO GITHUB, HEROKU, DOCKER, ETC.
%\capitulo{1}{Introduction}

\section{Context}

Music is a culturally significant and important element, with a wide variety of styles and genres.
The automated identification of musical styles is an emerging field of study, aiming to create intelligent systems capable of automatically recognizing and classifying the musical style of a song or music piece without any human intervention.

This field of study has important applications in areas such as music recommendation or digital audio library organization.

The application of artificial intelligence algorithms has shown considerable success in this area.

The present work aims to develop an automatic music style recognition system using audio processing and artificial intelligence algorithms.

\section{Project Structure}
TODO: ADD HERE PROJECT SECTIONS AND LINKS TO GITHUB, HEROKU, DOCKER, ETC.
\capitulo{2}{Objetivos del proyecto}
\section{Objetivos principales}
\begin{itemize}
\tightlist
\item Desarrollar un sistema de reconocimiento de estilos musicales utilizando inteligencia artificial.
\item Diseñar e implementar una aplicación web que permita usar el modelo de una forma sencilla.
\item Obtener conclusiones y conocimiento a partir de los datos.
\end{itemize}

\section{Objetivos técnicos}
\begin{itemize}
\tightlist
\item Desarrollar un sistema de \textit{aprendizaje automático} que, utilizando pistas de audio, detecte de forma automática el estilo musical basándose en diversas características.
\item Evaluar el modelo con diversas métricas como \textit{accuracy}, \textit{precision} o \textit{F1}.
\item Implementar tests unitarios y de integración.
\item Desarrollar un aplicación web utilizando librerías como Flask.
\item Desplegar la aplicación con el modelo entrenado a una plataforma online como Heroku.
\item Utilizar un sistema de control de versiones como Git para poder seguir los cambios en el código.
\item Seguir una metodología ágil como SCRUM para la gestión del desarrollo del proyecto.
\end{itemize}
%\capitulo{2}{Project objectives}

\section{Main objectives}
\begin{itemize}
\tightlist
\item Develop an automatic music style recognition system using artificial intelligence.
\item Design and implement a web application that uses the model.
\item Get insights about the data.
\end{itemize}

\section{Technical objectives}
\begin{itemize}
\tightlist
\item Develop a \textit{machine learning} model that using songs can detect the musical style automatically based on various features.
\item Evaluate machine learning model with various metrics and graphics.
\item Implement unitary and integration tests.
\item Develop a web application using a Python framework like Flask.
\item Deploy the application with the trained model online in a platform like Heroku.
\item Use a version control system as Git to track individual changes in the code.
\item Follow an Agile methodology as SCRUM to manage the project development.
\end{itemize}
\capitulo{3}{Conceptos teóricos}

Antes de empezar con el desarrollo del proyecto, es necesario explicar una serie de conceptos teóricos.

\section{Sonido}

El sonido es una vibración mecánica que se propaga a través de un medio elástico, como el aire, el agua o cualquier otro material. Estas vibraciones generan diferencias de presión en el medio, que son captadas por nuestros oídos y percibidas como sonido.

Matemáticamente, el sonido se puede representar mediante una función matemática $g(t)$, donde $t$ representa el tiempo. Esta función describe cómo varía la presión o desplazamiento de partículas en el medio a medida que pasa el tiempo.

\subsection{Representación Matemática del Sonido}

La representación matemática más común del sonido es la onda sinusoidal. Una onda sinusoidal se puede describir mediante la siguiente ecuación \cite{Benson_2007}:

\begin{equation}
g(t) = A \sin(2\pi ft + \phi)
\end{equation}

Donde:
\begin{itemize}
\item $A$ es la amplitud de la onda, que representa la máxima desviación de la onda desde su posición de equilibrio.
\item $f$ es la frecuencia de la onda, que determina la cantidad de ciclos completos que la onda realiza en un segundo.
\item $t$ es el tiempo.
\item $\phi$ es la fase inicial de la onda, que determina el desplazamiento horizontal de la onda en el tiempo.
\end{itemize}

\section{Reconocimiento musical}

El reconocimiento musical es un área que se centra en el análisis de las características del audio, para así, extraer información relevante. Por ejemplo la identificación de canciones, géneros musicales, o artistas.

\subsection{Características del Sonido}

El reconocimiento musical se basa en el análisis de diversas características del sonido. Algunas de las características más comunes son:

\begin{itemize}
\item \textbf{Ritmo}: El ritmo es una de las propiedades fundamentales de la música y se refiere a la organización temporal de los eventos sonoros. Su complejidad abarca desde un ritmo repetitivo a un ritmo con notas y silencios complejos.
El ritmo se puede analizar en términos de su complejidad, regularidad, velocidad y otros aspectos. En algunos géneros de música, como el techno, el ritmo suele ser el aspecto más dominante.
La importancia del ritmo en el reconocimiento musical puede involucrar la detección del tempo y el análisis de los patrones rítmicos para obtener conclusiones. \cite{Crossley-Holland}

\item \textbf{Frecuencia}: La frecuencia musical es el número de vibraciones u oscilaciones por segundo en el sonido. 
La frecuencia de estas vibraciones se mide en ciclos por segundo, o hercios (Hz). Mientras mayor sea la frecuencia más agudo es el tono, y cuando menor más grave.
La frecuencia musical se puede utilizar para identificar notas específicas o para determinar la armonía.

\item \textbf{Timbre}: El timbre se refiere a las características tonales y armónicas que distinguen diferentes instrumentos y voces. 
Dos instrumentos musicales pueden reproducir la misma nota pero sonar de una manera completamente diferente debido a su timbre.
El análisis del timbre puede ayudar a reconocer estilos musicales debido a que permite identificar los diferentes instrumentos implicados en una canción.

\item \textbf{Estructura Musical}: La estructura musical se refiere a la organización global de una composición musical.
Estudiando la estructura musical se pueden distinguir las diferentes partes de una canción como el estribillo, versos o secciones instrumentales. De esta manera se pueden encontrar patrones repetitivos y que se ajusten a estilos musicales concretos. 
Por ejemplo una gran cantidad de la música pop \cite{Team} tiene una estructura definida como: 

\begin{center}
\hfill \textbf{Verso} - \hfill \textbf{Estribillo} - \hfill \textbf{Verso} - \hfill \textbf{Estribillo} - \hfill \textbf{Puente} - \hfill \textbf{Estribillo}
\end{center}

\end{itemize}

Este tipo de características o patrones comunes pueden ser utilizados por modelos de aprendizaje automático para aprender de los datos y extraer conclusiones.

\subsection{Aplicaciones del Reconocimiento Musical}
El reconocimiento musical tiene diversas aplicaciones prácticas y de gran uso en la actualidad, algunas de las cuales son:

\begin{itemize}
\item \textbf{Recomendación de música}: Los algoritmos de reconocimiento musical se utilizan para recomendar música a los usuarios en función de sus preferencias y patrones de escucha. 
Estos sistemas analizan las características de las canciones más escuchadas por el usuario y crean un modelo de recomendación basado en sus gustos.

\item \textbf{Clasificación de géneros musicales}: El reconocimiento musical se utiliza para clasificar automáticamente las canciones en diferentes géneros musicales, lo que facilita la organización y la búsqueda de música en grandes bibliotecas digitales.
\end{itemize}

\section{Ejemplo teórico de extracción de características musicales}

\subsection{Espectrograma}
Un espectrograma es una representación visual del espectro de frecuencia de una señal de audio en función del tiempo. 
Proporciona información detallada sobre cómo se distribuye la energía del sonido en diferentes frecuencias a lo largo del tiempo.

A continuación se explica el proceso para obtener un espectrograma de una canción.

\begin{itemize}
\tightlist
\item \textbf{Preprocesamiento de la señal de audio}: La señal de audio de la canción se divide en segmentos. De esta manera es posible analizar la variación espectral en diferentes puntos de la señal a lo largo del tiempo.

\item \textbf{Transformada de Fourier de tiempo corto (STFT)}: Cada segmento de la señal se somete a una transformada de Fourier de tiempo corto (STFT). La STFT divide la señal en múltiples segmentos de tiempo y calcula la suma de diferentes frecuencias en cada segmento. 
Esto se logra mediante la aplicación de una ventana temporal a cada segmento y luego calculando la transformada de Fourier de cada ventana.

\item \textbf{Cálculo de la magnitud del espectro}: La STFT proporciona diversa información sobre las fases y amplitudes de las frecuencias en cada segmento de tiempo. Sin embargo, para construir un espectrograma, generalmente se toma la magnitud del espectro (amplitud absoluta de las frecuencias).

\item \textbf{Representación visual}: La magnitud del espectro se representa visualmente en un gráfico 2D, donde el eje horizontal representa el tiempo y el eje vertical representa las frecuencias. La intensidad del color o brillo en cada punto del gráfico indica la energía o amplitud de la frecuencia correspondiente.
\end{itemize}

Los espectrogramas tienen una gran utilidad no solo en el estudio de la música. Por ejemplo, se pueden utilizar para visualizar las señales de un radar y detectar objetos o en lingüística estudiando los patrones de frecuencia de los sonidos del habla y así estudiar los diferentes fonemas o acentos.

\imagen{example_spectrogram}{Espectrograma de una pista de audio.}{1.0}

\newpage

\subsection{Coeficientes Cepstrales de Frecuencias de Mel (MFCC)}
Los coeficientes cepstrales de frecuencias de Mel (MFCC) son características ampliamente utilizadas en el procesamiento de señales de audio y el reconocimiento de voz. 
Estos coeficientes representan las características espectrales de una señal de audio en función de la escala de Mel, que es una escala perceptual de frecuencia basada en la respuesta del oído humano. \cite{SAHIDULLAH2012543} \cite{Deruty_2022}

A continuación se explica el proceso para obtener los coeficientes MFCC de una pista de audio.

\begin{enumerate}
\tightlist
\item \textbf{Preénfasis}: La señal de audio se normaliza con un filtro de preénfasis, que resalta las altas frecuencias y compensa la atenuación de las más bajas. Esto ayuda a mejorar la relación señal-ruido y realzar las características relevantes en el espectro.

\item \textbf{División en tramas}: La señal preénfasis se divide en tramas o segmentos cortos y superpuestos en el tiempo. Esto se hace para capturar la variación espectral en diferentes puntos de la señal a lo largo del tiempo.

\item \textbf{Cálculo de la Transformada de Fourier de tiempo corto (STFT)}: A cada trama de la señal se le aplica la Transformada de Fourier de tiempo corto (STFT), que calcula la contribución de diferentes frecuencias en cada trama. La STFT proporciona diversa información sobre las fases y amplitudes de las frecuencias en cada segmento de tiempo.

\item \textbf{Banco de filtros de Mel}: Se aplica un banco de filtros de Mel, que consiste en una serie de filtros triangularmente espaciados en la escala de Mel. Estos filtros se utilizan para representar el espectro en términos de bandas de frecuencia de Mel.

\item \textbf{Logaritmo de la energía}: Se calcula el logaritmo de la energía después de aplicar el banco de filtros de Mel. Esto se hace para tener en cuenta la respuesta no lineal del oído humano a las frecuencias.

\item \textbf{Transformada de Coseno Discreta}: Se aplica la Transformada de Coseno Discreta (DCT) a los valores obtenidos anteriormente.

\item \textbf{Extracción de los coeficientes MFCC}: Finalmente, se seleccionan los coeficientes cepstrales más significativos para representar la información espectral de la señal de audio. Estos coeficientes son los utilizados como características para aplicaciones de procesamiento y reconocimiento de audio.
\end{enumerate}

Los coeficientes MFCC son ampliamente utilizados en aplicaciones como: 

\begin{itemize}
\tightlist
\item Reconocimiento de voz: Como los MFCC son una forma de reducir la complejidad de la voz a estructuras más simples y fácilmente procesables, son utilizados para el reconocimiento de voz. Por ejemplo asistentes de audio como Siri o Alexa utilizan (junto a otros sistemas) MFCCs. \cite{Kiran_2021}
\item Identificación de hablantes: Los MFCC se pueden utilizar para identificar voces de diferentes hablantes, debido a que cada persona tiene una forma característica de hablar. \cite{Kiran_2021} \cite{Nakagawa_Wang}
\item Clasificación de audio: Se pueden utilizar para identificar diferentes sonidos o instrumentos musicales en una pista de audio y de esta manera obtener información relevante que permita realizar una clasificación. \cite{Wu}
\end{itemize}

\imagen{example_MFCC}{MFCC de una pista de audio.}{1.0}

\newpage

\section{Inteligencia Artificial}

La inteligencia artificial (IA) es la capacidad de un sistema informático de imitar funciones cognitivas humanas, como el aprendizaje y la solución de problemas

Los sistemas de inteligencia artificial pueden analizar grandes cantidades de datos, reconocer patrones y tomar decisiones basadas en esa información. Pueden aprender de la experiencia y mejorar su rendimiento con el tiempo. 

Hay diferentes enfoques en la IA, incluyendo el aprendizaje automático (\textit{machine learning}), el procesamiento del lenguaje natural (\textit{natural language processing}), la visión por computador (\textit{computer vision}) y la robótica, entre otros.

La IA se utiliza en una amplia variedad de aplicaciones cómo en sistemas de recomendación, análisis de datos o diagnósticos médicos por ejemplo.

\subsection{Aprendizaje automático (Machine Learning)}
El aprendizaje automático (\textit{machine learning}) es un subcampo de la inteligencia artificial que se centra en el desarrollo de algoritmos y modelos que permiten a los sistemas aprender y extraer información a partir de datos, sin ser explícitamente programados para ello.
Existen diversos tipos de aprendizaje automático:

\begin{itemize}
\item \textbf{Aprendizaje supervisado}: Se proporciona como entrada a los algoritmos un conjunto de datos de entrenamiento etiquetados. El modelo aprende a realizar predicciones o tomar decisiones basadas en estos ejemplos etiquetados. El aprendizaje supervisado se utiliza en tareas de clasificación o regresión.

\item \textbf{Aprendizaje no supervisado}: Los algoritmos trabajan con conjuntos de datos no etiquetados, es decir, sin clase conocida. El objetivo es encontrar patrones u estructuras ocultas en los datos. El aprendizaje no supervisado se utiliza en tareas como el agrupamiento (\textit{clustering}) o la detección de valores atípicos (\emph{outliers}).

\item \textbf{Aprendizaje por refuerzo}: En este tipo de aprendizaje, un agente inteligente interactúa con su entorno y aprende a tomar decisiones según una serie de recompensas o penalizaciones. El objetivo es encontrar una política de actuación que maximize las recompensas recibidas. El aprendizaje por refuerzo es utilizado para entrenar agentes en videojuegos o en robótica. \cite{sutton2018reinforcement}

\item \textbf{Aprendizaje semi-supervisado}: En este tipo de aprendizaje el conjunto de datos no está completamente etiquetado, por lo tanto, el objetivo es maximizar el rendimiento del modelo a partir de los datos con clase conocida. \cite{chapelle2006ssl}
\end{itemize}

\section{Ejemplo de aprendizaje supervisado}

En este proyecto se va a utilizar un enfoque de IA utilizando aprendizaje supervisado, por lo que se va a detallar un ejemplo simple para entender su funcionamiento: \cite{DADA2019e01802}

\subsection{Objetivo}
Construir un modelo de aprendizaje automático para clasificar correos electrónicos como "<spam"> o "<no spam">.

\subsection{Conjunto de datos}
Conjunto de datos etiquetados que contiene 1000 correos electrónicos, donde cada correo tiene características como la frecuencia de ciertas palabras sospechosas de pertenecer a spam, la longitud del mensaje, la presencia de enlaces o imágenes, entre otros. \textbf{Además de incluir la clase a la que pertenece: "<Spam"> o "<No Spam">}.

\subsection{Preparación de los datos}
Se deben preparar los datos para el entrenamiento del modelo. 
Una opción adecuada es representar cada correo electrónico como un vector de características.

\begin{table}[ht]
\centering
\begin{tabular}{|c|c|c|c|c|}
\hline
\textbf{Correo} & \textbf{Palabras sospechosas} & \textbf{Longitud} & \textbf{Enlaces} & \textbf{Etiqueta} \\ \hline
1 & 1 & 120 & 0 & No spam \\
2 & 4 & 56 & 1 & Spam \\
3 & 1 & 352 & 2 & No spam \\
4 & 9 & 174 & 0 & Spam \\
\hline
\end{tabular}
\caption{Ejemplo de datos de entrenamiento en un problema de aprendizaje supervisado}
\end{table}

\subsection{Entrenamiento del modelo}
Una vez preparados los datos en un formato adecuado, estos son introducidos en un algoritmo de aprendizaje supervisado formando un modelo de aprendizaje automático. Por ejemplo Redes Neuronales Artificiales, Máquinas de Soporte Vectorial o Árboles de decisión.

\subsection{Predicción}
Una vez entrenado el modelo, se alimenta con datos externos y se realizan predicciones.

El algoritmo utilizado en el proyecto es más complejo y será explicado en detalle en las siguientes secciones.

\newpage

\section{Redes neuronales}
Para realizar el entrenamiento del modelo de aprendizaje automático se han utilizado redes neuronales. 
Las redes neuronales son una categoría de modelos de aprendizaje automático inspirados en la estructura biológica del cerebro humano. Se componen de nodos interconectados llamados neuronas, organizadas en capas, que trabajan juntas para procesar información y hacer predicciones.
La decisión de utilizar redes neuronales en lugar de otras técnicas de aprendizaje automático para el entrenamiento del modelo se basa en varios factores: \cite{9165253} \cite{Nandi_2021} 

\begin{itemize}
\tightlist

\item \textbf{Capacidad de modelado de características no lineales}: Partiendo sobre las bases asentadas a lo largo de este proyecto, las redes neuronales resultan ser especialmente útiles debido a la naturaleza compleja y no lineal de la música. Cada canción tiene numerosos componentes, como el ritmo, la melodía, la armonía, la instrumentación, el tempo, y el timbre, los cuales interactúan entre sí de maneras complejas para crear un género musical. Estas interacciones no son fácilmente modelables con técnicas lineales, y los géneros musicales no pueden ser claramente separados en base a solo una o dos características.

Las redes neuronales son capaces de entender estas relaciones y patrones no lineales al aprender representaciones internas de las canciones que capturan las características esenciales que definen cada género. Esto les permite realizar predicciones precisas del género musical, incluso en el caso de canciones que pueden fusionar elementos de varios géneros o que no se ajustan claramente a las definiciones de género establecidas.

\item \textbf{Capacidad para aprender automáticamente características}: En lugar de depender de la ingeniería manual de características, las redes neuronales son capaces de aprender representaciones útiles de los datos de entrada por sí mismas. Esto nos puede resultar particularmente beneficioso en tareas que requieren un alto nivel de comprensión de los datos, como el procesamiento de imágenes, texto o audio, como es nuestro caso. De este modo, estaremos utilizando una herramienta que en vez de depender de esta extracción manual de las características de las canciones, como los coeficientes de frecuencia mel, las características espectrales o el ritmo, es capaz de aprender automáticamente las representaciones más útiles a partir de los datos de audio brutos o preprocesados.

\item \textbf{Escalabilidad}: Las redes neuronales pueden manejar grandes volúmenes de datos y se benefician de la computación paralela, lo que las hace adecuadas para manejar las grandes cantidades de datos disponibles en muchos problemas modernos de aprendizaje automático. En el caso de la detección de géneros musicales, esta capacidad de las redes neuronales se vuelve imprescindible. Las bases de datos de canciones para entrenamiento y validación suelen ser enormes, comprendiendo miles o incluso millones de pistas, las cuales pueden ser eficientemente gestionadas a través de las redes neuronales, las cuales, partiendo de esta diversidad de ejemplos trabajarán para generar un modelo más robusto y preciso. 

\end{itemize}

Las redes neuronales se componen de los siguientes elementos: \cite{AGGARWAL_2023}
\begin{itemize}
\tightlist
\item \textbf{Neuronas}: entidad matemática inspirada en la neurona biológica. Es el componente básico de una red neuronal. 
\item \textbf{Capas}: las neuronas están organizadas por capas. Capa se refiere a un conjunto de neuronas que procesan información al mismo tiempo. Dependiendo de su posición en relación a los datos de entrada existen tres tipos de capas: capa de entrada, capas ocultas y capa de salida.
\item \textbf{Proceso de entrenamiento}: el proceso de entrenamiento de una red neuronal consiste en ajustar los pesos y sesgos de las neuronas de tal manera que el error entre las predicciones y los valores reales sea lo más pequeño posible. Existen diversos algoritmos para realizar este cálculo iterativo de los valores de cada neurona.
\end{itemize}

\subsection{Neuronas artificiales}
Una neurona artificial es una entidad matemática inspirada en una neurona biológica. Se utilizan para modelar la información que una red recibe, procesa y luego utiliza para tomar decisiones.
Al igual que las neuronas en el cerebro humano, una neurona artificial toma una serie de entradas y produce una salida. Cada entrada tiene un peso asociado, que ajusta el valor de esa entrada para la salida final. Los pesos y valores se ajustan durante el proceso de entrenamiento para mejorar el rendimiento de la red.

La neurona artificial suma las entradas ponderadas y luego aplica una función de activación para producir la salida. Las funciones de activación son necesarias para introducir no linealidad en la red, lo que permite que la red aprenda a partir de datos más complejos.

Hay varios tipos de neuronas diferenciadas por su función de activación. Algunos ejemplos podrían ser: \cite{Baheti} \cite{Gutpa_2022}

\textbf{Neurona umbral}: esta neurona aplica una función de umbral a las entradas. Si la suma ponderada de las entradas supera un cierto umbral, la neurona se activa.

La función de activación umbral se define de la siguiente manera:

\[
f(x) = \begin{cases}
    0, & \text{si } x < \text{umbral} \\
    1, & \text{si } x \geq \text{umbral}
\end{cases}
\]

Donde \text{umbral} es el valor de umbral que determina el punto de corte para la activación de la neurona.

\textbf{Neurona sigmoidal}: la neurona sigmoidal aplica una función sigmoide a las entradas

La función de activación sigmoidal se define de la siguiente manera:

\[
f(x) = \frac{1}{1 + e^{-x}}
\]

Donde $x$ es el valor de entrada de la neurona.

\textbf{Neurona ReLU \textit{(Rectified Linear Unit)}}: este tipo de neurona utiliza la función ReLU como su función de activación.
Este tipo de función de activación introduce no linealidad sin los problemas que pueden surgir con otras funciones de activación.

La función de activación ReLU se define de la siguiente manera:

\[
f(x) = \begin{cases}
    0, & \text{si } x < 0 \\
    x, & \text{si } x \geq 0
\end{cases}
\]

Donde $x$ es el valor de entrada de la neurona.

\textbf{Neurona softmax}: esta neurona se utiliza en la capa de salida para problemas de clasificación multiclase. La función Softmax toma un vector de entradas y produce un vector de salidas, cada una de las cuales está entre 0 y 1 y la suma total es 1. 
Cada salida puede interpretarse como la probabilidad de que la entrada pertenezca a una clase particular.

\[
f(x_i) = \frac{e^{x_i}}{\sum_{j=1}^{N} e^{x_j}}
\]

Donde $x_i$ es el valor de entrada de la neurona y $N$ es el número total de entradas.

\subsection{Capas}
Una red neuronal artificial está compuesta por un conjunto de capas neuronales interconectadas. \cite{AGGARWAL_2023} \cite{Ognjanovski_2020}
Cada capa consiste en un conjunto de neuronas las cuales reciben información de la capa anterior y envía su salida a las neuronas en la capa siguiente. Según su posición en la red neuronal las capas se pueden clasificar como:

\textbf{Capa de entrada}: es la primera capa de la red. Cada neurona en la capa de entrada corresponde a una característica en el conjunto de datos de entrada.

\textbf{Capas ocultas}: capas que se encuentran entre la capa de entrada y la capa de salida. Su número varía dependiendo de la profundidad de la red. 
En estas capas es donde se produce la mayor parte del cálculo de la red. Los pesos de las conexiones entre las neuronas en las capas ocultas se ajustan durante el entrenamiento.

\textbf{Capa de salida}: última capa de la red. Es la capa donde se devuelve la información. El número de neuronas en la capa de salida suele estar determinado por el tipo de problema que se está resolviendo. Por ejemplo, en un problema de clasificación binaria, solo se necesitaría una neurona de salida. En un problema de clasificación de varias clases, el número de neuronas de salida sería igual al número de clases.

\begin{figure}
\centering
\begin{tikzpicture}[scale=1.1]
  % Capa de entrada
  \foreach \x in {1,...,4}
    \node[circle, draw=black, fill=blue!20] (input\x) at (0,\x) {x\x};

  % Capa oculta 1
  \foreach \x in {1,...,5}
    \node[circle, draw=black, fill=green!20] (hidden1\x) at (2,\x) {h1\x};

  % Capa oculta 2
  \foreach \x in {1,...,3}
    \node[circle, draw=black, fill=green!20] (hidden2\x) at (4,\x+0.5) {h2\x};

  % Capa de salida
  \foreach \x in {1,...,2}
    \node[circle, draw=black, fill=red!20] (output\x) at (6,\x) {y\x};

  % Conexiones
  \foreach \i in {1,...,4}
    \foreach \j in {1,...,5}
      \draw[->] (input\i) -- (hidden1\j);

  \foreach \i in {1,...,5}
    \foreach \j in {1,...,3}
      \draw[->] (hidden1\i) -- (hidden2\j);

  \foreach \i in {1,...,3}
    \foreach \j in {1,...,2}
      \draw[->] (hidden2\i) -- (output\j);

  % Etiquetas de capas
  \node[align=center] at (0,-1) {Capa\\de entrada};
  \node[align=center] at (2,-1) {Capa\\oculta 1};
  \node[align=center] at (4,-1) {Capa\\oculta 2};
  \node[align=center] at (6,-1) {Capa\\de salida};

\end{tikzpicture}
\caption{Red neuronal con dos capas ocultas \cite{Wikilibros}}
\end{figure}

\subsection{Tipos de capas}

Las redes neuronales se componen de varias capas cada una con su funcionalidad específica. A continuación se exploran algunos tipos comunes de capas. \cite{Rosebrock_2023}

\textbf{Capas densas o totalmente conectadas}: cada neurona está conectada a todas las neuronas en la capa anterior y a todas las neuronas en la capa siguiente.

\textbf{Capas convolucionales}: capas formadas por redes neuronales convolucionales. En lugar de conectarse a todas las neuronas de la capa anterior, una neurona en una capa convolucional está conectada a un pequeño subconjunto de las neuronas en la capa anterior.

\textbf{Capas recurrentes}: capas formadas por redes neuronales recurrentes. En una capa recurrente, las neuronas tienen conexiones de retroalimentación con ellas mismas a lo largo del tiempo.

\textbf{Capas de pooling}: capas utilizadas para reducir la dimensionalidad de los datos. Una capa de pooling toma un subconjunto de las entradas y produce una sola salida, que es una medida estadística resumida de sus entradas (por ejemplo, el máximo o el promedio).

\textbf{Capas de normalización}: capas utilizadas para normalizar las entradas a una red o las salidas de una capa anterior. La normalización ayuda con el proceso de entrenamiento de la red.

\section{Proceso de entrenamiento}
El entrenamiento de una red neuronal implica un proceso que, de forma iterativa, ajusta los pesos de cada una de las neuronas para minimizar la diferencia entre las salidas predichas y las salidas reales del conjunto de datos de entrenamiento. Este proceso se puede clasificar en varios pasos: \cite{Pramoditha_2022}

\textbf{Inicialización de pesos}: el primer paso en el entrenamiento es inicializar los pesos de las neuronas de la red. Normalmente se inicializan de manera aleatoria.

\textbf{Propagación hacia adelante}: la propagación hacia adelante es el proceso por el cual la red neuronal transmite la información desde la capa de entrada hasta la capa de salida mediante la información calculada en cada función de activación neuronal.

\textbf{Función de perdida}: proceso de medición de la calidad de las predicciones. La función de perdida (o función de coste) genera un valor indicando cuanta distancia existe entre las predicciones y las clases verdaderas. Existen distintos tipos de funciones de perdida, en el caso de este proyecto la función utilizada es \textit{Sparse Categorical Crossentropy}, esta función es utilizada en problemas de clasificación multiclase dónde las etiquetas son datos enteros como es el caso del proyecto. \cite{Koech_2022}
La función es la siguiente:

\begin{equation}
L = - \log(y_{\mathrm{pred}}[y_{\mathrm{true}}])
\end{equation}

Donde:
\begin{itemize}
\tightlist
\item $y_{\mathrm{pred}}$ son las probabilidades predichas para cada clase.
\item $y_{\mathrm{true}}$ es la etiqueta verdadera.
\end{itemize}

\textbf{Backpropagation}: proceso de ajuste de los pesos de la red neuronal en función al error calculado en la función de perdida. El objetivo es minimizar iterativamente la función de pérdida propagando los errores desde la capa de salida hacia las capas ocultas por medio de la regla de la cadena, de esta manera se puede descubrir que número de neuronas deben actualizar sus pesos. La técnica que hace posible este proceso es el Descenso del Gradiente. 

\textbf{Descenso del Gradiente}: algoritmo de optimización utilizado para minimizar la función de pérdida. Su funcionamiento consiste en ajustar los parámetros del modelo de forma iterativa en la dirección negativa del gradiente de la función de pérdida. Su función es la siguiente:

\begin{equation}
\theta_{j} = \theta_{j} - \alpha \frac{\partial}{\partial \theta_{j}} J(\theta)
\end{equation}

Donde:
\begin{itemize}
\tightlist
\item $\theta_{j}$ es el peso a actualizar.
\item $J(\theta)$ es la función de pérdida.
\item $\frac{\partial}{\partial \theta_{j}} J(\theta)$ es el gradiente de la función de pérdida con respecto al parámetro $\theta_{j}$.
\item $\alpha$ es la tasa de aprendizaje o \textit{learning rate}, este parámetro controla cuanto varía $\theta_{j}$ en cada iteración.
\end{itemize}

\textbf{Iteración}: todo este proceso se repite $N$ veces. Cada proceso completo de inicialización, propagación hacia adelante, función de coste y backpropagation se llama época (\textit{epoch}). Ejecutar demasiadas épocas puede provocar sobreajuste, por este motivo, se implementan criterios de parada. Estos criterios pueden ser un número fijo de épocas, la estabilización de la función de pérdida (cuando no mejora tras un cierto número de épocas), una tasa de error constante en un conjunto de prueba, o si la disminución en la función de pérdida es menor que una tolerancia establecida.

\section{Evaluación del modelo}
La evaluación de un modelo de red neuronal es el proceso donde se determina la efectividad del modelo. Este proceso debe realizarse durante y después del entrenamiento.

\textbf{Evaluación con el conjunto de prueba}: tras entrenar el modelo es imprescindible realizar una evaluación final con el conjunto de prueba. Esta evaluación da como resultado una evaluación más precisa de como se comportará el modelo con datos externos.

\subsection{Cálculo de métricas de rendimiento}
Existen diversas métricas de rendimiento que indican como de bueno es el entrenamiento del modelo. Algunas de las más usadas en problemas de clasificación son:

\textbf{Accuracy}: proporción de predicciones correctas que hace el modelo en relación con todas las predicciones que hace.
\begin{equation}
\text{Accuracy} = \frac{\text{TP} + \text{TN}}{\text{TP} + \text{FP} + \text{FN} + \text{TN}}
\end{equation}

\textbf{Recall}: recall es la proporción de instancias positivas que el modelo predice correctamente en relación con todas las instancias positivas reales.
\begin{equation}
\text{Recall} = \frac{\text{TP}}{\text{TP} + \text{FN}}
\end{equation}

\textbf{F1-Score}: media armónica de precisión y recall.
\begin{equation}
\text{F1-Score} = 2 \cdot \frac{\text{Precision} \cdot \text{Recall}}{\text{Precision} + \text{Recall}}
\end{equation}

%\capitulo{3}{Theorical Concepts}

Before starting with the project development, it is necessary to explain a series of theoretical concepts.

\section{Sound}

Sound is a mechanical vibration propagating through an elastic medium, such as air, water or any other material. These vibrations generate some pressure differences in the medium, which are taken by our ears and perceived as sound.

Mathematically, sound can be represented by a mathematical function $f(t)$, where $t$ represents time. This function describes how the pressure of particles in the medium varies as time passes.

\subsection{Mathematical Representation of Sound}

The most common mathematical representation of sound is the sine wave. A sine wave can be described by the following equation:

\begin{equation}
f(t) = A \sin(2\pi ft + \phi)
\end{equation}

Donde:
\begin{itemize}
\item $A$ is wave amplitude, which represents the maximum deviation of the wave from its equilibrium position.
\item $f$ is wave frequency, which determines the number of complete cycles the wave makes in one second.
\item $t$ is the time.
\item $\phi$ is the initial wave phase, which determines the horizontal displacement of the wave over time.
\end{itemize}

\section{Music recognition}

Music recognition is an area that aims on the analysis of audio features in order to extract relevant information. For example, songs identification, musical genres, or artists.

\subsection{Sound features}

Music recognition is based on the analysis of diverse sound features. Some of the most common are:

\begin{itemize}
\item \textbf{Rythm}: Rhythm is a fundamental property of music and refers to the temporal organization of sound events. In music recognition it can involve tempo detection and analysis of musical patterns.

\item \textbf{Frequency}: Musical frequency is the number of vibrations or oscillations in the sound, per second. In music recognition, frequency spectrums can be analyzed to identify the musical notes.

\item \textbf{Timbre}: Refers to the tonal and harmonic characteristics that distinguish different instruments and voices. Music recognition can examine the timbre of an audio signal to identify the instruments used in a song for example.

\item \textbf{Music structure}: Music structure refers to the organization of a musical composition. In music recognition, structure analysis can detect changes and repetitions in the different parts of a song and identify specific musical styles.
\end{itemize}

\subsection{Musical Styles and Recognition}
Different musical styles often have distinctive characteristics that can be discovered in musical recognition. For example, certain genres have characteristic rhythms and harmonic patterns. 
These features can be identified using machine learning algorithms trained on a wide variety of labeled audio samples.

\subsection{Applications of Music Recognition}
Music recognition has several applications, some of which are:

\begin{itemize}
\item \textbf{Music recommendation}: Music recognition algorithms are used to recommend music to users based on their preferences and listening patterns. These systems analyze the features of the songs listened by the user and create a recommendation model based on their tastes.

\item \textbf{Genre classification}: Music recognition is used to automatically classify songs into different music genres, making it easier to organize and search for music in large digital libraries.
\end{itemize}

\section{Theorical examples of music features extractions}

\subsection{Spectrogram}
A spectrogram is a visual representation of the frequency spectrum of an audio signal as a function of time. It provides detailed information about how the sound is distributed over the different frequencies.

The process for obtaining a spectrogram of a song is explained below.

\begin{itemize}
\item \textbf{Audio signal preprocessing}: Audio signal is divided into segments, in a process called windowing. In this way, it is possible to analyze the spectral variation at different points of the signal over time.

\item \textbf{Short Time Fourier Transform (STFT)}: Each segment of the signal is applied to a short time Fourier transform (STFT). STFT divides the signal into multiple time segments and calculates the sum of different frequencies in each segment.  
This is accomplished by applying a time window to each segment and then calculating the Fourier transform of each window.

\item \textbf{Spectrum magnitude calculation}: The STFT provides various information about the phases and amplitudes of the frequencies in each time segment. However, to construct a spectrogram, only the magnitude of the spectrum (absolute amplitude of the frequencies) is taken.

\item \textbf{Visual representation}: Spectrum magnitude is represented visually in a 2D graph, where the horizontal axis represents time and the vertical axis represents frequencies. The intensity of the color or brightness at each point on the graph indicates the energy or amplitude of the corresponding frequency.
\end{itemize}

\imagen{example_spectrogram}{Audio track spectrogram.}{.5}

\subsection{Mel Frequency Cepstral Cepstral Coefficients (MFCC)}
Mel frequency cepstral coefficients (MFCCs) are widely used features in audio signal processing and speech recognition. 
These coefficients represent the spectral characteristics of an audio signal as a function of the Mel scale, which is a frequency scale based on the response of human ear.

The process for obtaining the MFCC coefficients of a song is explained below.

\begin{enumerate}
\item \textbf{Pre-emphasis}: Audio signal is normalized with a pre-emphasis filter, which highlights high-frequency frequencies and compensates for the attenuation of lower frequencies. This helps to improve the signal-to-noise ratio and enhance relevant features in the spectrum.

\item \textbf{Splitting}: Pre-emphasis signal is split into short, overlapping frames or segments over time. This is done to capture the spectral variation at different points in the signal over time.

\item \textbf{Short Time Fourier Transform (STFT)}: To each frame of the signal, the Short Time Fourier Transform (STFT) is applied, which calculates the contribution of different frequencies in each frame. The STFT provides various information about the phases and amplitudes of the frequencies in each time segment.

\item \textbf{Mel filter bank}: A Mel filter bank is applied, which consists of a series of triangularly spaced filters on the Mel scale. These filters are used to represent the spectrum in terms of Mel frequency bands.

\item \textbf{Logarithm of the energy}: This is done to include nonlinear response of the human ear to frequencies.

\item \textbf{Transformada de Coseno Discreta}: Discrete Cosine Transform (DCT) is applied to the values obtained above.

\item \textbf{Extraction of the MFCC coefficients}: Finally, the most significant cepstral coefficients are selected to represent the spectral information of the audio signal. These coefficients are the ones used as features for audio processing and recognition applications.
\end{enumerate}

MFCC coefficients are widely used in applications such as speech recognition, speaker identification and voice synthesis.

\imagen{example_MFCC}{Audio track MFCC.}{.5}

\section{Artificial Intelligence}

Artificial intelligence (AI) is the ability of a computer system to mimic human cognitive functions such as learning and problem solving.

Artificial intelligence systems can analyze large amounts of data, recognize patterns and make decisions based on that information. They can learn from experience and improve their performance.

There are different approaches to AI, including \textit{machine learning}, \textit{natural language processing}, \textit{computer vision} and robotics, among others.

AI is used in a wide variety of applications such as recommendation systems, data analysis or medical diagnostics, for example.

\subsection{Aprendizaje automático (Machine Learning)}
\textit{Machine learning} is a subfield of artificial intelligence that focuses on the development of algorithms and models that allow systems to learn and extract information from data, without being explicitly programmed to do so.
There are several types of machine learning:

\begin{itemize}
\item \textbf{Supervised learning}: A set of labeled training data is provided as input to the algorithms. The model learns to make predictions or decisions based on these labeled examples. Supervised learning is commonly used in classification or regression tasks.

\item \textbf{Unsupervised learning}: Algorithms work with unlabeled data sets, i.e. with no known class. The goal is to find hidden patterns or structures in the data. Unsupervised learning is used in tasks such as (\textit{clustering}).

\item \textbf{Reinforcement learning}: In this type of learning, an intelligent agent interacts with its environment and learns to make decisions according to a series of rewards or penalties. The goal is to find a policy that maximizes the rewards received. Reinforcement learning is used to train agents in video games or robotics.

\item \textbf{Semi-supervised learning}: In this type of learning the dataset is not completely labeled, therefore, the objective is to maximize the performance of the model from known data.
\end{itemize}

\section{Example of supervised learning}

In this project an AI approach using supervised learning is going to be used. Therefore, an example will be detailed in more detail:

\subsection{Objective}
Build a machine learning model to classify emails as "spam" or "non-spam".

\subsection{Dataset}
Labeled dataset containing 1000 emails, where each email has features such as the frequency of certain words suspected to belong to spam, message length, presence of links or images, for example. \textbf{In addition to including the class to which it belongs: "Spam" or "Not Spam"}.

\subsection{Data preparation}
The data must be prepared for model training. 
A suitable option is to represent each e-mail as a feature vector.

\begin{table}[ht]
\centering
\begin{tabular}{|c|c|c|c|c|}
\hline
\textbf{Email} & \textbf{Suspected spam words} & \textbf{Lenght} & \textbf{Links} & \textbf{Class} \\ \hline
1 & 1 & 120 & 0 & Not spam \\
2 & 4 & 56 & 1 & Spam \\
3 & 1 & 352 & 2 & Not spam \\
4 & 9 & 174 & 0 & Spam \\
\hline
\end{tabular}
\caption{Example of training data in a supervised learning problem}
\end{table}

\subsection{Model training}
Once the data is processed in a suitable format, it is fed into a supervised learning algorithm creating a machine learning model. For example, Artificial Neural Networks, Support Vector Machines or Decision Trees.

\subsection{Prediction}
Once the model is trained, it is fed with external data and predictions are made.

\begin{figure}[ht]
  \centering
  \setlength{\unitlength}{0.8cm}
  \begin{picture}(12,8)
    \put(4,7){\oval(5,2){\makebox(0,0){Email}}}
    \put(4,4.9){\oval(5,2){\makebox(0,0){Preprocessing}}}
    \put(4,2.8){\oval(5,2){\makebox(0,0){Feature vector}}}
    \put(4,0.7){\oval(5,2){\makebox(0,0){Training}}}
    \put(4,-1.4){\oval(5,2){\makebox(0,0){Prediction}}}
  \end{picture}
  \vspace{2cm}
  \caption{Example of supervised learning process}
\end{figure}

\capitulo{4}{Técnicas y herramientas}

Esta parte de la memoria tiene como objetivo presentar las técnicas metodológicas y las herramientas de desarrollo que se han utilizado para llevar a cabo el proyecto.

\section{Herramientas utilizadas}
\begin{itemize}
\item Python: Python es un lenguaje de programación de alto nivel ampliamente utilizado en el campo del aprendizaje automático y la inteligencia artificial. Posee una gran catidad de bibliotecas y frameworks que facilitan el procesamiento de datos y la implementación de algoritmos de inteligencia artificial.
En este proyecto ha sido utilizado para desarrollar toda la parte del back-end de la aplicación. 

\item librosa: librosa es una biblioteca de Python utilizada para el análisis y procesamiento de audio. Proporciona una amplia gama de métodos para extraer características de audio de forma sencilla. Algunos ejemplos son espectrogramas, coeficientes cepstrales de frecuencia mel (MFCC) o cromagramas.
En este proyecto ha sido utilizada para procesar y extraer diversas características de audio para alimentar los algoritmos de inteligencia artificial y crear el modelo.

\item TensorFlow: TensorFlow es una biblioteca de código abierto utilizada en el campo del aprendizaje automático. Proporciona una interfaz sencilla para la implementación de redes neuronales y otros algoritmos de aprendizaje automático.
TensorFlow ha sido la opción escogida para entrenar los algoritmos de redes neuronales del proyecto.

\item scikit-learn: scikit-learn es una biblioteca de aprendizaje automático de Python que proporciona una amplia gama de algoritmos y herramientas para el análisis de datos y la construcción de modelos. Incluye funciones para la división de conjuntos de datos en conjuntos de entrenamiento (\textit{train} y \textit{prueba}).

\item NumPy: NumPy es una biblioteca de Python utilizada para realizar cálculos numéricos en matrices y matrices multidimensionales. Proporciona una amplia gama de funciones matemáticas y herramientas para el manejo eficiente de datos numéricos.

\item Pandas: bibliteca de código abierto en Python que proporciona herramientas de análisis de datos. Es ampliamente usada en el mundo de la ciencia de datos.
\end{itemize}

\section{Justificación}

\subsection{Python}
A continuación se presentan algunas justificaciones que han decantado el desarrollo del proyecto en lenguaje Python en comparación con otros lenguajes de programación.

\textbf{Amplia variedad de bibliotecas y frameworks}: Python cuenta con una amplia gama de bibliotecas y frameworks especializados en aprendizaje automático, como TensorFlow, scikit-learn, PyTorch, Keras y muchos otros. 
Estas bibliotecas proporcionan herramientas e implementación de los principales algoritmos de inteligencia artificial, lo que facilita enormemente la tarea de desarrollar un proyecto de aprendizaje automático.

\textbf{Integración con otras tecnologías}: Python se integra fácilmente con otras tecnologías existentes en el ámbito del desarrollo del aprendizaje automático. Por ejemplo, se puede combinar con bases de datos (MySQL), herramientas de visualización (matplotlib), análisis de datos (Pandas) o en este caso herramientas de procesamiento musical (librosa).

\subsection{TensorFlow vs scikit-learn}
La elección de utilizar TensorFlow en lugar de scikit-learn se basa en varias consideraciones.

\textbf{En primer lugar}, TensorFlow es adecuado para proyectos que involucran aprendizaje profundo y redes neuronales. Proporciona una mayor cantidad de herramientas para crear, entrenar y desplegar modelos de aprendizaje profundo que scikit-learn. 

\textbf{TensorFlow} permite implementar arquitecturas de redes neuronales complejas, como redes neuronales convolucionales (CNN), las cuales son muy adecuadas para los objetivos del proyecto.

\textbf{Por último}, TensorFlow tiene soporte para utilizar tarjetas gráficas (GPUs) a la hora de realizar el proceso de entrenamiento. Esto proporciona una ventaja significativa en rendimiento, ya que al poder utilizar la gran cantidad de nucleos de una GPU se pueden realizar cálculos de manera paralela y distribuida lo que resulta en tiempos
de entrenamiento considerablemente más rápidos en comparación con el uso exclusivo de CPU.

\section{Técnicas utilizadas}

Se han utilizado diversas técnicas y metodologías para llevar a cabo el desarrollo y entrenamiento de los modelos de aprendizaje automático.

(En el documento \textit{Anexos} se describe de una forma mucho más profunda el desarrollo metodológico del proyecto.)

\section{Procesamiento y extracción de características de audio}

Para el procesamiento y extracción de características de audio, se ha utilizado la biblioteca librosa en Python.

\begin{itemize}
\item MFCC (coeficientes cepstrales de frecuencia mel): Estos coeficientes capturan las características espectrales del audio y se utilizan comúnmente en tareas de reconocimiento de voz y clasificación de audio. Son las características principales que se han utilizado para entrenar el modelo y realizar predicciones.
\end{itemize}

\section{División del conjunto de datos}

Para dividir el conjunto de datos en conjuntos de entrenamiento, prueba y validación, se ha utilizado la biblioteca scikit-learn. 
Scikit-learn proporciona una función llamada \textit{train\textunderscore test\textunderscore split} que permite dividir el conjunto de datos en partes destinadas para el entrenamiento y la evaluación del modelo. Esta técnica de división del conjunto de datos es fundamental para evaluar la capacidad de \textbf{generalización} del modelo y evitar el \textbf{sobreajuste}.

\begin{figure}
  \centering
  \begin{tikzpicture}
    % Ancho de los conjuntos de entrenamiento y prueba
    \def\datasetwidth{6}
    
    % dataset
    \draw[fill=gray!30] (-1,0) rectangle (\datasetwidth,2);
    
    % test
    \draw[fill=red!30] (-1,0.4) rectangle (1.5,1.6);
    \node at (0,1) {\textbf{Test}};
    
    % train
    \draw[fill=blue!30] (1.5,0.4) rectangle (\datasetwidth,1.6);
    \node at (3.75,1) {\textbf{Train}};
    
    \draw[dashed] (1.5,0) -- (1.5,2);
    
    \draw[dashed] (-1,0.4) -- (\datasetwidth,0.4);
    \draw[dashed] (-1,1.6) -- (\datasetwidth,1.6);
    
    % Etiqueta del conjunto completo
    \node at (\datasetwidth/2,-0.5) {\textbf{Conjunto de datos}};
  \end{tikzpicture}
  \caption{División del conjunto de datos}
\end{figure}

\section{Redes neuronales con TensorFlow}

Para realizar el entrenamiento de los datos, se han utilizado redes neuronales implementadas con la biblioteca TensorFlow.

En este proyecto, se han utilizado diferentes arquitecturas de redes neuronales, como redes neuronales convolucionales (CNN) y redes neuronales multicapa. Estas arquitecturas son adecuadas para tareas de procesamiento de audio y han demostrado ser efectivas en la clasificación y reconocimiento de patrones en audio.

Las redes neuronales se entrenan actualizando los pesos y los sesgos de la red iterativamente para minimizar la pérdida (\textit{loss}). Una vez entrenadas, las redes neuronales realizan predicciones sobre nuevos datos de audio, clasificándolos en categorías o estilos musicales.

\begin{figure}
  \centering
  \begin{tikzpicture}[
    neuron/.style={circle, draw, minimum size=1.5cm},
    connection/.style={->, thick}
  ]
    % Input layer
    \foreach \i in {1,2,3}
      \node[neuron] (input\i) at (0,-\i*1.5) {};

    % Hidden layer 1
    \foreach \i in {1,2,3,4}
      \node[neuron] (hidden1\i) at (3,-\i*1.5) {};

    % Hidden layer 2
    \foreach \i in {1,2,3,4}
      \node[neuron] (hidden2\i) at (6,-\i*1.5) {};

    % Output layer
    \node[neuron] (output) at (9, -3) {};

    % Connections
    \foreach \i in {1,2,3}
      \foreach \j in {1,2,3,4}
        \draw[connection] (input\i) -- (hidden1\j);

    \foreach \i in {1,2,3,4}
      \foreach \j in {1,2,3,4}
        \draw[connection] (hidden1\i) -- (hidden2\j);

    \foreach \i in {1,2,3,4}
      \draw[connection] (hidden2\i) -- (output);

    \node[above] at (0,0) {Capa de entrada};
    \node[above] at (3,0) {Capa oculta 1};
    \node[above] at (6,0) {Capa oculta 2};
    \node[above] at (9,0) {Capa de salida};

  \end{tikzpicture}
  \caption{Ejemplo de una red neuronal multicapa}
\end{figure}

\section{Desarrollo con Python y Flask}

La parte del desarrollo de la aplicación web se ha llevado a cabo utilizando Python y el framework Flask. Flask es un framework web que permite construir aplicaciones web en Python.

Utilizando Flask, se ha desarrollado el back-end de la aplicación, que se encarga de recibir las solicitudes del usuario, procesar los datos de audio y devolver los resultados de clasificación y análisis. Todo ello mediante la implementación de RESTful API, lo que proporciona una comunicación contínua entre el usuario y el servidor.


%\capitulo{4}{Techniques and tools}

The objective of this part of the project is to present the methodological techniques and development tools used in the project's development.

\section{Tools}
\begin{itemize}
\item Python: Python is a high-level programming language widely used in the field of machine learning and artificial intelligence. It has a high number of libraries and frameworks that facilitate data processing and implementation of artificial intelligence algorithms.
In this project it has been used to develop all the back-end part of the application.

\item librosa: librosa is a Python library used for audio analysis and processing. It provides a wide range of methods to extract audio features in a simple way. Some examples are spectrograms, mel frequency cepstral coefficients (MFCC) or chromagrams.
In this project it has been used to process and extract various audio features to feed the artificial intelligence algorithms and create the model.

\item TensorFlow: TensorFlow is an open source library used in the field of machine learning. It provides a simple interface for the implementation of neural networks and other machine learning algorithms.
TensorFlow has been the chosen option to train the neural network algorithms of the project.

\item scikit-learn: scikit-learn is a Python machine learning library that provides a wide range of algorithms and tools for data analysis and model building. Includes functions for splitting datasets into \textit{train} and \textit{test}.

\item NumPy: NumPy is a Python library used to perform numerical calculations on matrices and multidimensional arrays. It provides a wide range of mathematical functions and tools for efficient handling of numerical data.

\item Pandas: open source Python library that provides data analysis tools. It is very used in data science fields.
\end{itemize}

\section{Justification}

\subsection{Python}
The following includes some of the justifications that have led to the development of the project in Python language compared to other languages.

\textbf{Extensive variety of libraries and frameworks}: Python has a broad range of libraries and frameworks specialized in machine learning, such as TensorFlow, scikit-learn, PyTorch, Keras and many others. 
These libraries provide tools and implementation of the main artificial intelligence algorithms, which facilitates the task of developing a machine learning project.

\textbf{Integration with other technologies}: Python easily integrates with other existing technologies in machine learning development area. For example, it can be combined with relational databases (MySQL), visualization tools (matplotlib, Seaborn), data analysis (Pandas) or in this case, music processing tools (librosa).

\subsection{TensorFlow vs scikit-learn}
The choice to use TensorFlow instead of scikit-learn is based on several considerations.

\textbf{First}, TensorFlow is suitable for projects involving deep learning and neural networks. It provides a higher number of tools for creating, training, and deploying deep learning models than scikit-learn. 

\textbf{TensorFlow} allows implementing complex neural network architectures, such as convolutional neural networks (CNNs), which are particularly well suited for this project.

\textbf{Finally}, TensorFlow supporst graphics cards (GPUs) to perform the training process. This provides a very significant performance advantage, as the large number of cores in a GPU can be used to perform parallel and distributed computations. The result is a considerably faster training time compared to a GPU.

\section{Techniques}

Several techniques and methodologies have been used to develop and train the machine learning models.

(A much more detailed description of the methodological development of the project can be found in the document \textit{Annexes}).

\section{Processing and extracting audio features}

librosa has been used for the audio processing and features extraction.

\begin{itemize}
\item MFCC (mel frequency cepstral coefficients): These coefficients capture the spectral features of audio and are commonly used in speech recognition and audio classification tasks. They are the main features that have been used to train the model and make predictions.
\end{itemize}

\section{Dataset splitting}

scikit-learn library has been used to split the dataset into training, test and validation data.
Scikit-learn provides a function called \textit{train\textunderscore test\textunderscore split} that splits the dataset into parts intended for training and model evaluation. 
This data set splitting technique is critical to evaluate the model's generalizability and to avoid overfitting.

\begin{figure}
  \centering
  \begin{tikzpicture}
    % Width of training and test sets
    \def\datasetwidth{6}
    
    % dataset
    \draw[fill=gray!30] (-1,0) rectangle (\datasetwidth,2);
    
    % test
    \draw[fill=red!30] (-1,0.4) rectangle (1.5,1.6);
    \node at (0,1) {\textbf{Test}};
    
    % train
    \draw[fill=blue!30] (1.5,0.4) rectangle (\datasetwidth,1.6);
    \node at (3.75,1) {\textbf{Train}};
    
    \draw[dashed] (1.5,0) -- (1.5,2);
    
    \draw[dashed] (-1,0.4) -- (\datasetwidth,0.4);
    \draw[dashed] (-1,1.6) -- (\datasetwidth,1.6);
    
    % Etiqueta del conjunto completo
    \node at (\datasetwidth/2,-0.5) {\textbf{Dataset}};
  \end{tikzpicture}
  \caption{Dataset splitting}
\end{figure}

\section{Neural networks with TensorFlow}

Neural networks implemented with TensorFlow library have been used to train the data.

In this project, various neural network architectures have been used, such as convolutional neural networks (CNN) and multilayer neural networks. These architectures are suitable for audio processing tasks and have proven to be effective in audio classification and pattern recognition.

Neural networks are trained by updating the weights and biases of the network iteratively to minimize \textit{loss}. Once trained, neural networks make predictions on new audio data, classifying them into categories or musical styles.

\begin{figure}
  \centering
  \begin{tikzpicture}[
    neuron/.style={circle, draw, minimum size=1.5cm},
    connection/.style={->, thick}
  ]
    % Input layer
    \foreach \i in {1,2,3}
      \node[neuron] (input\i) at (0,-\i*1.5) {};

    % Hidden layer 1
    \foreach \i in {1,2,3,4}
      \node[neuron] (hidden1\i) at (3,-\i*1.5) {};

    % Hidden layer 2
    \foreach \i in {1,2,3,4}
      \node[neuron] (hidden2\i) at (6,-\i*1.5) {};

    % Output layer
    \node[neuron] (output) at (9, -3) {};

    % Connections
    \foreach \i in {1,2,3}
      \foreach \j in {1,2,3,4}
        \draw[connection] (input\i) -- (hidden1\j);

    \foreach \i in {1,2,3,4}
      \foreach \j in {1,2,3,4}
        \draw[connection] (hidden1\i) -- (hidden2\j);

    \foreach \i in {1,2,3,4}
      \draw[connection] (hidden2\i) -- (output);

    \node[above] at (0,0) {Input layer};
    \node[above] at (3,0) {Hidden layer 1};
    \node[above] at (6,0) {Hidden layer 2};
    \node[above] at (9,0) {Output layer};

  \end{tikzpicture}
  \caption{Multi-layer neural network example}
\end{figure}

\section{Development with Python and Flask}

Web application development part was implemented using Python and the Flask framework. Flask is a web framework that allows building web applications in Python.

The back-end of the application has been developed in Flask, which is responsible for receiving user requests, processing audio data and returning classification and analysis results. 
All this through the implementation of RESTful API, which provides continuous communication between the user and the server.


\capitulo{5}{Aspectos relevantes del desarrollo del proyecto}

\section{Desarrollo del Proyecto}

El desarrollo de este proyecto se ha llevado a cabo utilizando una metodología ágil basada en Scrum. Este enfoque proporciona flexibilidad necesaria para adaptar cambios en los requerimientos y para mejorar iterativamente el proyecto durante su desarrollo.
Las metodología ágiles se basan en la idea de dividir el proyecto en ciclos iterativos llamados "sprints", donde se llevan a cabo actividades de planificación, desarrollo, pruebas y revisión. 
Estos sprints suelen tener una duración fija, en este caso han sido de 1 semana con alguna excepción justificada.

\subsection{Metodología Scrum}
En el documento \textit{Anexos} se explica en detalle el proceso de desarrollo del proyecto.

En resumen el desarrollo ha consistido en los siguientes pasos iterativos:

\begin{enumerate}
\item \textbf{Planificación inicial}: Etapa de definición de objetivos generales del proyecto. Creación del Product Backlog.

\item \textbf{Reuniones de Sprint}: Al comienzo de cada sprint, se realiza una pequeña reunión para revisar las tareas realizadas y seleccionar las tareas a desarrollar.

\item \textbf{Desarrollo del sprint}: Durante el sprint, se trabaja en cada una de las tareas asignadas.

\item \textbf{Mejora continua}: La metodología ágil promueve la mejora continua y el aprendizaje a lo largo del desarrollo del proyecto.
\end{enumerate}

\subsection{Análisis}

El objetivo del proyecto es aplicar técnicas de Inteligencia Artificial (IA) y Aprendizaje Automático (ML) para la clasificación de estilos musicales. 
Durante la fase de análisis, se estudiaron varios modelos de ML, eligiendo finalmente la Red Neuronal Convolucional (CNN) debido a su mayor eficacia.

Otro factor a tener en cuenta en un proyecto de IA es el conjunto de datos. Se han pensado diversos conjuntos de datos para entrenar el modelo como:
\begin{itemize}

\item \textbf{GTZAN Dataset}: conjunto de datos muy utilizado en la clasificación musical. Recopilado por George Tzanetakis en 2002, consta de \textbf{1000 fragmentos} de audio de \textbf{30 segundos}, distribuidos en \textbf{10 géneros musicales}.

\item \textbf{Million Song Dataset}: conjunto de datos formado por las \textit{características musicales y metadatos} de un millón de canciones populares. No incluye pistas de audio.

\item \textbf{MagnaTagATune}: conjunto de datos formado por \textbf{25863 fragmentos} de audio de \textbf{29 segundos}.

\item \textbf{FMA (Free Music Archive)}: conjunto de datos musicales, libre de derechos, constituido por \textbf{106574 fragmentos} de audio de \textbf{30 segundos}. 

\end{itemize}

Finalmente se ha elegido el conjunto de datos \textbf{FMA (Free Music Archive) de 7.2 GiB}, con 8000 fragmentos de audio para realizar el entrenamiento del modelo. Esto es debido a que ofrece una gran cantidad de pistas musicales libres de derechos. Además, al incluir las pistas musicales (a diferencia de Million Song Dataset por ejemplo)
es posible extraer las características de forma manual utilizando librosa u otras herramientas.

\subsection{Diseño}

En cuanto al diseño de la aplicación, en un primer momento se pensó en el desarrollo de una aplicación de escritorio. Sin embargo, esta idea fue descartada y se optó por desarrollar una aplicación web debido a las siguientes razones:

\begin{itemize}

\item \textbf{Evolución tecnológica}: Es una realidad que las aplicaciones web están dominando el momento tecnológico actual, por lo que realizar el proyecto en web es una buena manera de estar al día de esta tecnología.

\item \textbf{Accesibilidad}: La aplicación web es accesible desde cualquier dispositivo con una conexión a internet, independientemente del sistema operativo. De esta manera se amplía el alcance de la aplicación.

\item \textbf{Actualizaciones}: Cuando se actualiza una aplicación web, automaticamente los usuarios reciben la última versión disponible en producción, eliminando la necesidad de descargar e instalar actualizaciones manualmente. 

\item \textbf{Mantenimiento}: Generalmente es más sencillo mantener una aplicación web que una aplicación de escritorio, ya que factores como hardware o sistema operativo del usuario desaparecen.
\end{itemize}

El marco de trabajo escogido para realizar la implementación de la aplicación web ha sido:

\begin{itemize}

\item \textbf{Backend}: Para el backend se ha utilizado Flask, un marco de trabajo de Python, pensado para realizar aplicaciones web y APIs.

\item \textbf{Frontend}: Para el frontend, se ha optado por HTML y CSS. HTML es un lenguaje de marcado que define la estructura y contenidos de las páginas web que conforman la aplicación. CSS es utilizado para definir los estilos de los documentos HTML y añadir elementos atractivos para el usuario.

\end{itemize}

\newpage

\section{Extracción de características de audio}

En este sección se va a analizar el proceso de extracción de características de audio.

\subsection{Extracción de MFCC usando Python y librosa}

El método de extracción de MFCC está diseñado para recorrer todos los archivos en un directorio especificado y extraer las características MFCC de cada archivo de audio que encuentre. La explicación de los detalles relevantes es la siguiente:

\begin{verbatim}
	y, sr = librosa.load(item)

	if librosa.get_duration(y=y, sr=sr) < 25:
		continue

	y = y[:(25 * sr)]
	mfcc = librosa.feature.mfcc(y=y, sr=sr, n_mfcc=10)
	mfcc_normalized = (mfcc - np.mean(mfcc)) / np.std(mfcc)

	mfccs[int(item.stem)] = mfcc_normalized.ravel()
\end{verbatim}

\begin{enumerate}
\item \textbf{y, sr = librosa.load(item)}: Carga del archivo de audio, devolviendo tanto la señal de audio (y) como la frecuencia de muestreo (sr).

\item \textbf{if librosa.get\textunderscore duration (y=y, sr=sr) < 25}: Comprobación de la duración del audio. Si el audio es inferior a 25 segundos se omite el archivo y se continúa con el siguiente.

\item \textbf{y = y[:(25 * sr)]}: Recorte de los primeros 25 segundos de la señal de audio.

\item\textbf{mfcc = librosa.feature.mfcc(y=y, sr=sr, n\textunderscore mfcc =10)}: Extracción de 10 coeficientes MFCC.

\item \textbf{mfcc\textunderscore normalized = (mfcc - np.mean(mfcc)) / np.std(mfcc)}: Normalización de los coeficientes MFCC restando la media y dividiendo por la desviación estándar.

\item\textbf{mfccs[int(item.stem)] = mfcc\textunderscore normalized.ravel()}: Los coeficientes MFCC normalizados se aplanan en una matriz 1D y se almacenan en un diccionario.
\end{enumerate}

\subsection{Extracción del resto de características}

Para comparar el rendimiento y elegir el mejor conjunto de características de audio posibles para realizar la clasificación, se ha repetido el proceso con:

\begin{enumerate}
\item \textbf{Espectrogramas}: librosa.feature.melspectrogram(y=y, sr=sr)

\item \textbf{Cromagramas}: librosa.feature.chroma\textunderscore stft(y=y, sr=sr)
\end{enumerate}

\newpage

\section{Implementación de las redes neuronales}

En esta sección se van a estudiar las redes neuronales utilizadas en el proyecto y su eficacia.

\subsection{Modelo de Red Neuronal inicial}

\begin{verbatim}
initial_model = tf.keras.Sequential([
        tf.keras.layers.Dense(512, activation='relu'),
        tf.keras.layers.Dropout(0.2),
        tf.keras.layers.Dense(512, activation='relu'),
        tf.keras.layers.Dropout(0.2),
        tf.keras.layers.Dense(512, activation='relu'),
        tf.keras.layers.Dense(164, activation='softmax')
])
\end{verbatim}

Primer modelo genérico que se utilizó para estudiar la calidad del entrenamiento y sus predicciones.
Este modelo específico es un ejemplo de una red neuronal totalmente conectada, que se configura como una secuencia de capas.

\texttt{tf.keras.layers.Dense(512, activation='relu')}

Primera capa del modelo. Cada neurona en esta capa está conectada a todas las neuronas en la capa anterior. Tiene 512 neuronas y usa la función de activación ReLU (Rectified Linear Unit). La función \textbf{ReLU} es una función de activación no líneal por lo que el modelo puede aprender patrones complejos.

\texttt{tf.keras.layers.Dropout(0.2)}

Capa de Dropout. Dropout es una técnica de regularización utilizada para evitar el sobreajuste. Durante el entrenamiento, aleatoriamente se desactivan algunas neuronas para evitar un ajuste excesivo a los datos de entrenamiento. En este caso se desactivan el 20\% de las neuronas.

\texttt{tf.keras.layers.Dense(512, activation='relu')} 

\texttt{tf.keras.layers.Dropout(0.2)}:

Estas son la segunda capa oculta y la segunda capa de Dropout, respectivamente.

\texttt{tf.keras.layers.Dense(512, activation='relu')}

Esta es la tercera capa oculta del modelo. Al igual que las dos primeras capas ocultas, tiene 512 neuronas y utiliza la función de activación ReLU.

\texttt{tf.keras.layers.Dense(164, activation='softmax')}

Capa de salida del modelo. Tiene 164 neuronas, que corresponden al número de estilos musicales presentes en el dataset. La función de activación \textbf{Softmax} produce una distribución de la probabilidad entre las diferentes clases.

TODO: INTRODUCIR GRÁFICOS CON LOS RESULTADOS DEL ENTRENAMIENTO Y REPRESENTACIÓN DE LAS CAPAS

\newpage

\subsection{Modelo de Red Neuronal Convolucional 1D (CNN 1D)}

\begin{verbatim}
conv_model = tf.keras.Sequential([
    tf.keras.layers.Reshape((10770, 1), input_shape=(None, 10770)),
    tf.keras.layers.Conv1D(64, 3, padding='same', activation='relu'),
    tf.keras.layers.Conv1D(64, 3, padding='same', activation='relu'),
    tf.keras.layers.BatchNormalization(),
    tf.keras.layers.MaxPooling1D(pool_size=2),

    tf.keras.layers.Conv1D(128, 3, padding='same', activation='relu'),
    tf.keras.layers.Conv1D(128, 3, padding='same', activation='relu'),
    tf.keras.layers.BatchNormalization(),
    tf.keras.layers.MaxPooling1D(pool_size=2),

    tf.keras.layers.Conv1D(256, 3, padding='same', activation='relu'),
    tf.keras.layers.Conv1D(256, 3, padding='same', activation='relu'),
    tf.keras.layers.BatchNormalization(),
    tf.keras.layers.MaxPooling1D(pool_size=2),

    tf.keras.layers.Conv1D(128, 3, padding='same', activation='relu'),
    tf.keras.layers.Conv1D(128, 3, padding='same', activation='relu'),
    tf.keras.layers.BatchNormalization(),
    tf.keras.layers.MaxPooling1D(pool_size=2),

    tf.keras.layers.Flatten(),
    tf.keras.layers.Dense(512, activation='relu'),
    tf.keras.layers.Dropout(0.5),
    tf.keras.layers.Dense(512, activation='relu'),
    tf.keras.layers.Dense(164, activation="softmax")
])
\end{verbatim}

Las redes neuronales convolucionales son especialmente eficaces para la clasificación musical debido al reconocimiento de patrones locales, en el contexto de la música es especialmente relevante ya que puede traducirse en la capacidad de identificar patrones como ritmo o tonos de forma eficaz.

\texttt{tf.keras.layers.Reshape((10770, 1), input\_shape=(None, 10770))}

Primera capa del modelo. Se cambia la forma de los datos de entrada a un formato aceptado por las capas convolucionales propuestas. En este caso, los datos de entrada se reforman a una matriz 2D de 10770 filas y 1 columna.

\texttt{tf.keras.layers.Conv1D(64, 3, padding='same', activation='relu')}
\texttt{tf.keras.layers.Conv1D(64, 3, padding='same', activation='relu')}

Primeras capa convolucionales. La convolución se realiza con \textbf{64 filtros} y un \textbf{tamaño de kernel de 3}. La función de activación es la ReLU (Rectified Linear Unit).

\texttt{tf.keras.layers.BatchNormalization()}

Capa de normalización por lotes. Técnica necesaria para estabilizar y acelerar el proceso de entrenamiento. El funcionamiento consiste en aplicar una transformación que mantenga la salida media cercana a 0 y la desviación típica cercana a 1.

\texttt{tf.keras.layers.MaxPooling1D(pool\_size=2)}

Capa de pooling. Reduce la dimensión espacial de la entrada extrayendo las características más importantes y así prevenir el sobreajuste.

Posteriormente se realizan los mismos pasos con diferentes capas de 128 neuronas, 256 neuronas y finalmente otras 128 neuronas. Hasta llegar a las últimas capas.

\texttt{tf.keras.layers.Flatten()}

Esta capa, reduce la dimensionalidad de la entrada a una entrada 1D. \textbf{Es el puente entre las capas convolucionales y las capas densas.}

\texttt{tf.keras.layers.Dense(512, activation='relu')}

Capa completamente conectada de 512 neuronas con una función de activación ReLU.

\texttt{tf.keras.layers.Dropout(0.5)}

Capa de Dropout que desactiva aleatoriamente el 50\% de las neuronas para evitar el sobreajuste.

\texttt{tf.keras.layers.Dense(512, activation='relu')}

Capa completamente conectada de 512 neuronas con una función de activación ReLU.

\texttt{tf.keras.layers.Dense(164, activation="softmax")}

Capa de salida del modelo. Tiene 164 neuronas, que corresponden al número de estilos musicales presentes en el dataset. La función de activación \textbf{Softmax} produce una distribución de la probabilidad entre las diferentes clases.

Este modelo es una CNN 1D con una estructura formada por una serie de bloques de capas convolucionales seguidas de normalización y pooling, seguidos por una serie de capas densas.

La primera parte del modelo (las capas convolucionales) se encarga de aprender características locales en pequeñas ventanas de los datos de entrada, mientras que la segunda parte del modelo, formado por capas completamente conectadas, aprende a combinar estas características para hacer predicciones.

TODO: INTRODUCIR GRÁFICOS CON LOS RESULTADOS DEL ENTRENAMIENTO
%\capitulo{5}{Relevant aspects of the project development}

\section{Project Development}

The development of this project has been conducted using an agile methodology based on Scrum. This approach provides flexibility to adapt to changes in requirements and to iteratively improve the project during its development.
Agile methodologies are based on the idea of splitting the project into iterative cycles called "sprints", where planning, development, testing and review activities are done.
Sprints usually have a fixed duration, in this case they have been 1 week each, with some justified exceptions.

\subsection{Scrum Methodology}
The project development process is explained in detail in the document \textit{Annexes}.

In summary, the development consisted of the following iterative steps:

\begin{enumerate}
\item \textbf{Initial Phase}: Definition of the project's general objectives. Creation of the Product Backlog.

\item \textbf{Sprint planning meetings}: At the beginning of each sprint, a small meeting is held to review the tasks performed and select the tasks to be developed.

\item \textbf{Sprint development}: During the sprint, work is done on each of the assigned tasks.

\item \textbf{Continuous improvement}: Agile methodology promote continuous improvement throughout the development of the project.
\end{enumerate}

\subsection{Analysis}

The objective of the project is to apply Artificial Intelligence (AI) and Machine Learning (ML) techniques for classification of musical styles.
During the analysis phase, several ML models were studied, to finally choosing the Convolutional Neural Network (CNN) due to its higher efficiency.

Another factor to take into account in an AI project is the dataset. Various datasets have been thought of to train the model such as:
\begin{itemize}

\item \textbf{GTZAN Dataset}: dataset widely used in music classification. Compiled by George Tzanetakis in 2002, it consists of \textbf{1000 audio fragments} of \textbf{30 seconds}, distributed in \textbf{10 musical styles}.

\item \textbf{Million Song Dataset}: dataset formed of \textit{musical features and metadata} of one million popular songs. It does not include audio tracks.

\item \textbf{MagnaTagATune}: dataset consisting of \textbf{25863 audio fragments} of \textbf{29 seconds}.

\item \textbf{FMA (Free Music Archive)}: royalty-free music dataset formed of \textbf{106574 audio fragments} of \textbf{30 seconds}. 

\end{itemize}

Finally, \textbf{FMA (Free Music Archive) dataset of 7.2 GiB}, with 8000 audio fragments has been chosen to perform the training of the model. This is because it offers a large amount of royalty-free music tracks. 
In addition, by including the music tracks (unlike Million Song Dataset for example) it is possible to extract the features manually using librosa or other tools.

\subsection{Design}

Regarding the design of the application, at first the development of a desktop application was considered. However, this idea was abandoned and it was opted to develop a web application due to the following reasons:

\begin{itemize}
\item \textbf{Technological evolution}: It is a reality that web applications are leading the current technological moment, so making the project on the web is a good way learn about this technology.

\item \textbf{Accessibility}: The web application is accessible from any device with an internet connection, regardless of the operating system. In this way, the scope of the application is extended.

\item \textbf{Updates}: When a web application is updated, users automatically receive the latest version available in production, avoiding the need to download and install updates manually. 

\item {Maintenance}: It is generally easier to maintain a web application than a desktop application, as factors such as the user's hardware or operating system disappear.
\end{itemize}

The framework chosen for the implementation of the web application was:

\begin{itemize}

\item \textbf{Backend}:For the backend the selected option is Flask, a Python framework, designed for web applications and APIs.

\item \textbf{Frontend}: For the frontend, HTML and CSS have been chosen. HTML is a markup language that defines the structure and contents of the web pages that form the application. CSS is used to define the styles of HTML documents and add attractive elements for the user.

\end{itemize}

\newpage

\section{Extraction of audio features}

This section will discuss the audio feature extraction process.

\subsection{MFCC extraction using Python and librosa}

MFCC extraction method is designed to loop through all files in a specified directory and extract the MFCC features of each audio file it finds. The explanation of the relevant details is as follows:

\begin{verbatim}
	y, sr = librosa.load(item)

	if librosa.get_duration(y=y, sr=sr) < 25:
		continue

	y = y[:(25 * sr)]
	mfcc = librosa.feature.mfcc(y=y, sr=sr, n_mfcc=10)
	mfcc_normalized = (mfcc - np.mean(mfcc)) / np.std(mfcc)

	mfccs[int(item.stem)] = mfcc_normalized.ravel()
\end{verbatim}

\begin{enumerate}
\item \textbf{y, sr = librosa.load(item)}: Load audio file. Returns both the audio signal (y) and the sample rate (sr).

\item \textbf{if librosa.get\textunderscore duration (y=y, sr=sr) < 25}: Audio length check. If the audio is shorter than 25 seconds, the file is skipped.

\item \textbf{y = y[:(25 * sr)]}: Trims the audio signal to the first 25 seconds.

\item\textbf{mfcc = librosa.feature.mfcc(y=y, sr=sr, n\textunderscore mfcc =10)}: Extraction of 10 MFCC coefficients.

\item \textbf{mfcc\textunderscore normalized = (mfcc - np.mean(mfcc)) / np.std(mfcc)}: Normalization of MFCC coefficients by subtracting the mean and dividing by the standard deviation.

\item\textbf{mfccs[int(item.stem)] = mfcc\textunderscore normalized.ravel()}: The normalized MFCC coefficients are flattened into a 1D matrix and stored in a dictionary.
\end{enumerate}

\subsection{Extraction of other characteristics}

To compare performance and select the best possible set of audio features to perform the classification, the process was repeated with:

\begin{enumerate}
\item \textbf{Spectrograms}: librosa.feature.melspectrogram(y=y, sr=sr)

\item \textbf{Chromagrams}: librosa.feature.chroma\textunderscore stft(y=y, sr=sr)
\end{enumerate}

\newpage

\section{Implementation of neural networks}

In this section we will study the neural networks used in the project and their effectiveness.

\subsection{Initial Neural Network Model}

\begin{verbatim}
initial_model = tf.keras.Sequential([
        tf.keras.layers.Dense(512, activation='relu'),
        tf.keras.layers.Dropout(0.2),
        tf.keras.layers.Dense(512, activation='relu'),
        tf.keras.layers.Dropout(0.2),
        tf.keras.layers.Dense(512, activation='relu'),
        tf.keras.layers.Dense(164, activation='softmax')
])
\end{verbatim}

First generic model used to study the quality of training and its predictions.
This specific model is an example of a fully connected neural network, which is configured as a sequence of layers.

\texttt{tf.keras.layers.Dense(512, activation='relu')}

First layer of the model. Each neuron in this layer is connected to all neurons in the previous layer. It has 512 neurons and uses the ReLU (Rectified Linear Unit) activation function. The \textbf{ReLU} function is a non-linear activation function so the model can learn complex patterns.

\texttt{tf.keras.layers.Dropout(0.2)}

Dropout Layer. Dropout is a regularization technique used to avoid overfitting. During training, some neurons are randomly deactivated to avoid overfitting to the training data. In this case, 20\% of the neurons are deactivated.

\texttt{tf.keras.layers.Dense(512, activation='relu')} 

\texttt{tf.keras.layers.Dropout(0.2)}:

These are the second hidden layer and the second Dropout layer, respectively.

\texttt{tf.keras.layers.Dense(512, activation='relu')}

This is the third hidden layer of the model. Like the first two hidden layers, it has 512 neurons and uses the ReLU activation function.

\texttt{tf.keras.layers.Dense(164, activation='softmax')}

Output layer of the model. It has 164 neurons, matching the number of musical styles present in the dataset. The activation function \textbf{Softmax} produces a probability distribution among the different classes.

TODO: ENTER GRAPHS WITH TRAINING RESULTS AND LAYER REPRESENTATION

\newpage

\subsection{1D Convolutional Neural Network Model (1D CNN)}

\begin{verbatim}
conv_model = tf.keras.Sequential([
    tf.keras.layers.Reshape((10770, 1), input_shape=(None, 10770)),
    tf.keras.layers.Conv1D(64, 3, padding='same', activation='relu'),
    tf.keras.layers.Conv1D(64, 3, padding='same', activation='relu'),
    tf.keras.layers.BatchNormalization(),
    tf.keras.layers.MaxPooling1D(pool_size=2),

    tf.keras.layers.Conv1D(128, 3, padding='same', activation='relu'),
    tf.keras.layers.Conv1D(128, 3, padding='same', activation='relu'),
    tf.keras.layers.BatchNormalization(),
    tf.keras.layers.MaxPooling1D(pool_size=2),

    tf.keras.layers.Conv1D(256, 3, padding='same', activation='relu'),
    tf.keras.layers.Conv1D(256, 3, padding='same', activation='relu'),
    tf.keras.layers.BatchNormalization(),
    tf.keras.layers.MaxPooling1D(pool_size=2),

    tf.keras.layers.Conv1D(128, 3, padding='same', activation='relu'),
    tf.keras.layers.Conv1D(128, 3, padding='same', activation='relu'),
    tf.keras.layers.BatchNormalization(),
    tf.keras.layers.MaxPooling1D(pool_size=2),

    tf.keras.layers.Flatten(),
    tf.keras.layers.Dense(512, activation='relu'),
    tf.keras.layers.Dropout(0.5),
    tf.keras.layers.Dense(512, activation='relu'),
    tf.keras.layers.Dense(164, activation="softmax")
])
\end{verbatim}

Convolutional neural networks are particularly effective for music classification due to local pattern recognition, in the context of music it is especially relevant as it can translate into the ability to identify patterns such as rhythm or tones efficiently in the different parts of the songs.

\texttt{tf.keras.layers.Reshape((10770, 1), input\_shape=(None, 10770))}

First layer of the model. Input data is reshaped to a format accepted by the proposed convolutional layers. In this case, the input data is reshaped to a 2D matrix of 10770 rows and 1 column.

\texttt{tf.keras.layers.Conv1D(64, 3, padding='same', activation='relu')}
\texttt{tf.keras.layers.Conv1D(64, 3, padding='same', activation='relu')}

First convolutional layers. The convolution is performed with \text{64 filters} and a \text{kernel size of 3}. The activation function is the ReLU (Rectified Linear Unit).

\texttt{tf.keras.layers.BatchNormalization()}

Batch normalization layer. Important step involved in the stabilization and acceleration of the training process. The operation consists of applying a transformation that keeps the mean output close to 0 and the standard deviation close to 1.

\texttt{tf.keras.layers.MaxPooling1D(pool\_size=2)}

Pooling layer. Reduces the spatial dimension of the input by extracting the most important features and preventing overfitting.

Subsequently, the same steps are performed with different layers of 128 neurons, 256 neurons and finally another 128 neurons. Until the last layers are reached.

\texttt{tf.keras.layers.Flatten()}

This layer reduces the dimensionality of the input to a 1D input. \textbf{Is the bridge between the convolutional layers and the dense layers}.

\texttt{tf.keras.layers.Dense(512, activation='relu')}

Fully connected layer of 512 neurons with a ReLU activation function.

\texttt{tf.keras.layers.Dropout(0.5)}

Dropout layer that randomly deactivates 50\% of the neurons to avoid overfitting.

\texttt{tf.keras.layers.Dense(512, activation='relu')}

Fully connected layer of 512 neurons with a ReLU activation function.

\texttt{tf.keras.layers.Dense(164, activation="softmax")}

Output layer of the model. It has 164 neurons, matching the number of musical styles present in the dataset. The activation function \textbf{Softmax} produces a probability distribution among the different classes.

This model is a 1D CNN with a structure formed by a series of blocks of convolutional layers followed by normalization and pooling, followed by a series of dense layers.

The first part of the model (the convolutional layers) is responsible for learning local features in small windows of the input data, while the second part of the model, fully connected layers, learns to combine these features to make predictions.

TODO: ENTER GRAPHS WITH TRAINING RESULTS AND LAYER REPRESENTATION
\capitulo{6}{Trabajos relacionados}

\section{Estado del arte}

El estado del arte es una expresión utilizada para describir el estado más alto de desarrollo en un cierto campo en el momento actual. 
En este apartado se va a investigar y exponer cuál es el método o el estudio más avanzado y con mayor acierto a la hora de clasificar géneros musicales.

\subsection{Estudios recientes}



\section{Trabajos relacionados}

\begin{itemize}
\tightlist

\item \textbf{Zero-shot Learning for Audio-based Music Classification and Tagging} \cite{choi2020zeroshot}: El artículo explora el uso del \textit{zero-shot learning} en la clasificación musical. 
Se aborda el problema de los datos sin etiquetar utilizando un espacio semántico adicional de etiquetas para descubrir la relación entre ellas. 
Los resultados de este estudio son prometedores y genera conclusiones y líneas futuras de trabajo como el usar las letras de las canciones para obtener información relevante sin intervención humana.

\item \textbf{Detecting Music Genre Using Extreme Gradient Boosting} \cite{10.1145/3184558.3191822}: Estudio en el que se utilizan diversos métodos de clasificación para predecir géneros musicales. Se trata de un estudio muy completo ya
que utiliza diversos métodos como CNN (\textit{convolutional neural network}), DNN (\textit{deep neural network}), árboles de decisión y ensembles (\textit{SGBoost, ExtraTrees}).

\item \textbf{Environmental sound classification using temporal-frequency attention based convolutional neural network} \cite{Mu_Yin_Huang_Xu_Du_2021}: Artículo que explora el uso de TFCNN (temporal-frequency attention based convolutional neural network model)
para clasificar sonidos ambientales. Es interesante observar los resultados, con una tasa de acierto superior al 90\% en casi todos los escenarios.
\end{itemize}

\capitulo{7}{Conclusiones y Líneas de trabajo futuras}

\section{Conclusiones}
A través de este proyecto, se ha demostrado la eficacia pero también la complejidad del uso de la Inteligencia Artificial (IA) para la clasificación automática de estilos musicales. 

En líneas generales, el sistema ha demostrado tener una eficacia razonable en la tarea de clasificar estilos musicales. No obstante, en el proceso también se han identificado limitaciones.

En primer lugar, aunque el proceso de extracción de características de audio juega un papel crucial en el desarrollo y entrenamiento de los modelos de clasificación de estilos musicales, podría ser conveniente considerar la exploración de otros métodos o la implementación de técnicas más sofisticadas con el objetivo de enriquecer la calidad de los datos que alimentan a la red neuronal.

Podría ser beneficioso explorar técnicas de \textit{deep learning} como el aprendizaje por transferencia (\textit{transfer learning}), los modelos \textit{pre-entrenados} podrían usarse como un punto de partida, lo que podría mejorar la eficiencia del aprendizaje y potencialmente mejorar la precisión del modelo.

La cantidad del conjunto de datos tiene un impacto directo en la eficacia del sistema. Utilizar un conjunto de datos más extenso y variado podría mejorar la capacidad del sistema para generalizar.

\textbf{En conclusión}, este proyecto ha servido para demostrar que la clasificación de estilos musicales utilizando IA es factible y que la inteligencia artificial aplicada a la música ofrece un gran potencial y motivación para investigaciones futuras.

\section{Líneas de trabajo futuras}

\begin{enumerate}
\item \textbf{Experimentación con diferentes modelos}: Se han utilizado principalmente redes neuronales convolucionales en este proyecto, pero hay muchas otras arquitecturas de red. Por ejemplo, las redes neuronales recurrentes (RNN) o modelos generativos adversarios (GANs) podrían ser buenas opciones para experimentar.

\item \textbf{Explorar enfoques de aprendizaje semi-supervisado y no supervisado}: Dada la dificultad de etiquetar grandes conjuntos de datos musicales, también se podría considerar la utilización de técnicas de aprendizaje semi-supervisado o no supervisado. Estos enfoques podrían ser útiles para aprovechar datos no etiquetados y mejorar el rendimiento del sistema.

\item \textbf{Incorporación de feedback del usuario}: Permitir a los usuarios corregir las clasificaciones incorrectas del modelo y utilizar esa información para mejorar el modelo.

\item \textbf{Construcción de un sistema de recomendación}: Teniendo el modelo de clasificación de estilos musicales, podría ser interesante construir un sistema de recomendación musical que sugiera canciones a los usuarios basándose en sus preferencias y hábitos de escucha.
\end{enumerate}

TODO: ADD MORE FUTURE ENHANCEMENTS AND IMPROVE THIS SECTION!
%\capitulo{7}{Conclusions and future works}

\section{Conclusions}
Through this project, the performance but also the complexity of using Artificial Intelligence (AI) for the automatic classification of musical styles has been proved.

In general terms, the system has proven to be reasonably effective in the task of classifying musical styles. However, limitations have also been identified in the process.

First, although the audio feature extraction process plays a crucial role in the development and training of musical style classification models, it might be worth considering to explore other methods or implementing more advanced techniques in order to improve the quality of the data that feeds the neural network.

It may be beneficial explore deep learning techniques such as transfer learning. Pre-trained models could be used as a starting point, which could improve the quality of the model.

The size of the dataset has a direct impact on the effectiveness of the system. Using a larger and more varied dataset could improve the system's ability to generalize.

\textbf{In conclusion}, this project has served to demonstrate that music style classification using AI is feasible and that artificial intelligence applied to music offers great potential for future research.

\section{Future works}

\begin{enumerate}
\item \textbf{Experimentation with different models}: Convolutional neural networks have been mainly used in this project, but there are many other network architectures. For example, recurrent neural networks (RNNs) or generative adversarial models (GANs) could be fine choices for experimentation.

\item \textbf{Explore semi-supervised and unsupervised learning approaches}: Given the difficulty of labeling large music datasets, semi-supervised or unsupervised learning techniques could be considered. These approaches could be useful to take advantage of unlabeled data and improve system performance.

\item \textbf{User feedback}: Allow users to correct model incorrect classifications and use that information to improve the model.

\item \textbf{Build a recommendation model}: With the music style classification model, it could be interesting to build a music recommendation system that suggests songs to the users, based on their preferences and listening habits.
\end{enumerate}

TODO: ADD MORE FUTURE ENHANCEMENTS AND IMPROVE THIS SECTION!


\bibliographystyle{plain}
%\bibliography{bibliografia}

\end{document}
